\section {előadás (2025. szeptember 16.)}
Könyv: \url{../konyv.pdf#page=108}
\subsubsection*{Ismétlés}
\textbf{Gale-Shapley algoritmus: } tetszőleges preferencialisták esetén generál (egy) stabil párosítást.\\
\textbf{Megjegyzés: } a stabil párosítások száma akár exponenciálisan nagy lehet (n-ben).\\
\textbf{Megjegyzés: } időnként egész "egzotikus" stabil párosítások is előfordulnak, pl. az összes fiú/lány a preferencialistájának első/utolsó lányát/fiúját kapja.

\subsubsection*{Optimalitás}

Egy adott fiú és adott lány \textbf{szóba jönnek egymásnak}, ha van olyan stabil párosítás, amelyben ők egy párt alkotnak.\\
Egy adott fiú és adott lány \textbf{lelki társa} egymásnak, ha minden stabil párosításban ők egy párt alkotnak.\\

\textbf{Állítás}
\begin{itemize}
    \item \textbf{A: } az algoritmus minden fiúhoz a számára szóba jövő lányok közül a neki legjobban tetszőt párosítja.
    \item \textbf{B: } algoritmus minden lányhoz a számára szóba jövő fiúk közül a neki legkevésbé tetszőt párosítja.
\end{itemize}

\textbf{Bizonyítás}\\
\textbf{A)}

Indirekt tegyük fel, hogy van olyan fiú, akihez a Gale-Shapley algoritmus a számára szóba jövő lányok közül nem a neki legjobban tetszőt párosítja.
Ekkor ez a fiú valamelyik nap szerenádozik a számára szóba jövő lányok közül a neki legjobban tetszőnél, akitől kosarat kap.
Tekintsük azt a napot, amikor egy ilyen szomorú esemény először fordul elő.
Legyen a kosarat kapó fiú $f$ és a kosarat kapó lány $l$, és $f'$ az a fiú, aki miatt $f$ kosarat kapott.

Ekkor $l$-nek jobban tetszik $f'$, mint $f$.
Mivel az első olyan napon vagyunk, amikor egy fiút kikosaraz a számára szóba jövő lányok közül a neki legjobban tetsző, $f'$-t nem kosarazhatta még ki a számára szóba jövő lányok közül a neki legjobban tetsző $l^*$.
Egy $l^*$ nem lehet $l$ előtt $f'$ preferencialistáján ($l^*=l$ lehetséges).

Ezek után tekintsünk egy olyan $M$ stabil párosítást, amelyben $f$ és $l$ egy párt alkotnak.
Ez különbözik a Gale-Shapley algoritmus által meghatározott párosítástól.
Jelölje $l''$ az $f'$ $M$-beli párját. Mivel $l''$ szóba jön $f'$ számára, $l''$ nem lehet $l^*$ előtt $f'$ preferencialistáján ($l''=l^*$ lehetséges).

\textbf{Konklúzió: } $f' \rightarrow (l, l^*, l'')$, azaz $f'$-nek jobban tetszik $l$, mint $l''$

\textbf{Illusztráció ($M$-ben a párok):}

\fbox{\begin{tikzpicture}
\node(f){f};
\node(l)[right = 0.5cm of f]{l};
\node(f')[below = 0.5cm of f]{f'};
\node(l'')[below = 0.5cm of l]{l''};
\draw (f) -- (l);
\draw (f') -- (l'');
\draw[dashed] (f') -- (l);
\end{tikzpicture}}

Az eredeti feltevésünk szerint $f'$ és $l$ instabilitás ($M$-re) $\rightarrow$ \textbf{ellentmondás}.

\textbf{B)}

Indirekt tegyük fel, hogy van olyan lány, akihez a Gale-Shapley algoritmus a számára szóba jövő fiúk közül nem a neki legkevésbé tetszőt párosítja.
Legyen $l$ egy ilyen lány.
Legyen $M$ egy olyan stabil párosítás, amelyben $l$ párja $f$, kevésbé tetszik $l$-nek, mint a Gale-Shapley algoritmus szolgáltatta $f'$ párja.
Legyen $f'$ $M$-beli párja $l'$.

\textbf{Illusztráció ($M$-ben a párok):}

\fbox{\begin{tikzpicture}
\node(f){f};
\node(l)[right = 0.5cm of f]{l};
\node(f')[below = 0.5cm of f]{f'};
\node(l')[below = 0.5cm of l]{l'};
\draw (f) -- (l);
\draw (f') -- (l');
\end{tikzpicture}}

Most a feltételünk szerint $l$-nek jobban tetszik $f'$, mint $f$.
Másrészt a Gale-Shapley algoritmus $f'$-höz a szóba jövő lányok közül a neki legjobban tetszőt párosítja, következésképpen $f'$-nek jobban tetszik $l$ (Gale-Shapley-beli párja), mint $l'$ ($M$-beli párja).
Egy $f'$ és $l$ instabilitás ($M$-re) $\rightarrow$ \textbf{ellentmondás}.

\subsubsection*{Egyértelműség}
Futtassuk le a Gale-Shapley algoritmust úgy, hogy a fiúk szerenádoznak $\rightarrow M_1$.\\
Ezután futtassuk le úgy, hogy a lányok szerenádoznak $\rightarrow M_2$.

$M_1$ és $M_2$ stabil párosítások.\\
Ha $M_1 \neq M_2$, akkor van legalább két különböző stabil párosítás.\\
Ha $M_1 = M_2$, akkor ez az egyetlen stabil párosítás.
Mivel ez egyben fiú-optimális és fiú-pesszimális, valamint lány-optimális és lány-pesszimális, mindenki számára pontosan egy ellenkező nemű jön szóba.

\subsubsection*{Stabil szobatárs probléma (unisex változat)}
$2n$ fiú van. Mindenkinek van egy preferencialistája a többiekről.
2 ágyas szobákba szeretnénk beosztani őket.\\
Instabilitás: $\left[f_i-f_j\right]$ és $\left[f_k-f_l\right]$\\
$f_i$ szívesebben lenne $f_k$-val egy szobában, mint $f_j$-vel.\\
$f_k$ szívesebben lenne $f_i$-vel egy szobában, mint $f_l$-lel.\\
Nem feltétlenül van stabil párosítás.
