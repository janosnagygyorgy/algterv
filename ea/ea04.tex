\section{előadás (2025. szeptember 30.)}
\subsubsection*{Gyors mátrixszorzás}
Könyv: \url{konyv.pdf#page=12}

\textbf{Feladat:} $\underset{p \times q}{A}$ és $\underset{q \times r}{B}$ kompatibilis mátrixok AB szorzatának kiszámítása.

\begin{tikzpicture}[
    squarenode/.style={rectangle, draw=black, minimum width=1cm, minimum height=1cm},
]
\node[squarenode](A){$A$};
\node[squarenode](B)[above right = 0.1cm and 0.1cm of A]{$B$};
\node[squarenode](C)[right = 0.1cm of A]{$C$};
\node(A_left)[above left = -0.25cm and -0.05cm of A]{};
\node(A_right)[above right = -0.25cm and -0.05cm of A]{};
\node(B_top)[above left = -0.05cm and -0.25cm of B]{};
\node(B_bottom)[below left = -0.05cm and -0.25cm of B]{};
\draw[-Latex] (A_left) -- (A_right);
\draw[-Latex] (B_top) -- (B_bottom);
\end{tikzpicture}

$\underset{p \times r}{C} = \underset{p \times q}{A} \cdot \underset{q \times r}{B}$\\
$c_{ij} = \overset{q}{\underset{k=1}{\sum}} a_{ik} \cdot b_{kj} \rightarrow$ 1 elem kiszámításához $q$ db elemi szorzás és $q-1$ db elemi összeadás kell.\\
$p \times r$ darab elem $\rightarrow pqr$

Ha ezek mind $n \times n$-es mátrixok:
\begin{itemize}
    \item $M(n) = n^3 \in \Theta(n^3)$ (M az elemi szorzások száma)
    \item $S(n) = n^3 \in \Theta(n^3)$ (S az elemi összeadások száma)
\end{itemize}
Cél: ennél hatékonyabb algoritmus.

Ha A, B, C nem négyzetes, 0-ákkal feltöltjük:\\
\begin{tikzpicture}[
    squarenode/.style={rectangle, draw=black, minimum size=1.5cm},
    rectanglenode/.style={rectangle, draw=black, minimum width=1.3cm, minimum height=1cm}
]
\node[squarenode](square){};
\node[rectanglenode](rect)[above left = -1.015cm and -1.315cm of square]{A};
\node[left = 0cm of square.west]{$n$};
\node[above = 0cm of square.north]{$n$};
\end{tikzpicture}
\begin{tikzpicture}[
    squarenode/.style={rectangle, draw=black, minimum size=1.5cm},
    rectanglenode/.style={rectangle, draw=black, minimum width=1cm, minimum height=1.3cm}
]
\node[squarenode](square){};
\node[rectanglenode](rect)[above left = -1.315cm and -1.015cm of square]{B};
\node[left = 0cm of square.west]{$n$};
\node[above = 0cm of square.north]{$n$};
\end{tikzpicture}

Oszd meg és uralkodj ötlet:\\
\begin{tikzpicture}[
    squarenode/.style={rectangle, draw=black, minimum size=0.7cm},
]
\node[squarenode](a11){$A_{11}$};
\node[squarenode](a12)[right = 0cm of a11]{$A_{12}$};
\node[squarenode](a21)[below = 0cm of a11]{$A_{21}$};
\node[squarenode](a22)[right = 0cm of a21]{$A_{22}$};
\node[left = 0cm of a11]{$\frac{n}{2}$};
\node[above = 0cm of a11]{$\frac{n}{2}$};
\node[above = 0cm of a12]{$\frac{n}{2}$};
\node[left = 0cm of a21]{$\frac{n}{2}$};
\node[left = 0.7cm of $(a11)!0.5!(a21)$]{$A=$};
\end{tikzpicture}
\begin{tikzpicture}[
    squarenode/.style={rectangle, draw=black, minimum size=0.7cm},
]
\node[squarenode](b11){$B_{11}$};
\node[squarenode](b12)[right = 0cm of b11]{$B_{12}$};
\node[squarenode](b21)[below = 0cm of b11]{$B_{21}$};
\node[squarenode](b22)[right = 0cm of b21]{$B_{22}$};
\node[left = 0.5cm of $(b11)!0.5!(b21)$]{$B=$};
\end{tikzpicture}
\begin{tikzpicture}[
    squarenode/.style={rectangle, draw=black, minimum size=0.7cm},
]
\node[squarenode](c11){$C_{11}$};
\node[squarenode](c12)[right = 0cm of c11]{$C_{12}$};
\node[squarenode](c21)[below = 0cm of c11]{$C_{21}$};
\node[squarenode](c22)[right = 0cm of c21]{$C_{22}$};
\node[left = 0.5cm of $(c11)!0.5!(c21)$]{$C=$};
\end{tikzpicture}

Az $A$ és $B$ mátrixok helyett 4-4 db $\frac{n}{2} \times \frac{n}{2}$-es mátrix.
\begin{flalign*}
    C_{11} &= A_{11} \cdot B_{11} + A_{12} \cdot B_{21}&&\\
    C_{12} &= A_{11} \cdot B_{12} + A_{12} \cdot B_{22}&&\\
    C_{21} &= A_{21} \cdot B_{11} + A_{22} \cdot B_{12}&&\\
    C_{22} &= A_{21} \cdot B_{12} + A_{12} \cdot B_{22}
\end{flalign*}

Elemi szorzások száma:
$
M(n) =
\begin{cases}
    1 & n=1\\
    8M\left(\frac{n}{2}\right) & n \geq 2\\
\end{cases}
$  

Elemi összeadások száma:
$
S(n) =
\begin{cases}
    0 & n=1\\
    8S\left(\frac{n}{2}\right) + 4\left(\frac{n}{2}^2\right) & n \geq 2\\
\end{cases}
$

\textbf{Kérdés: hatékonyabb-e ez a módszer?}

Csökkent-e az $M(n)$ és $S(n)$ érték?
Használjuk a Mester tételt: $T(n) = a \cdot T\left(\frac{n}{b}\right) + f(n)$

\begin{itemize}
    \item $M(n) = 8M\left(\frac{n}{2}\right)$
    \item $a = 8$, $b = 2$, $f(n) = 0$
    \item $n^{\log_2 8} = n^3$
\end{itemize}

M.T. 1. eset: ha $f(n) \in \mathcal{O}(n^{log_b a} - \mathcal{E})$ valamilyen $\mathcal{E} > 0$-ra (pl. $\mathcal{E}=0.5$), akkor $M(n) \in \Theta(n^3)$.

\begin{itemize}
    \item $S(n) = 8M\left(\frac{n}{2}\right) + n^2$
    \item $a = 8$, $b = 2$, $f(n) = n^2 \in \mathcal{O}(n^{\log_2 8 - \mathcal{E}})$, pl. $\mathcal{E}=0.5$
    \item $n^{\log_2 8} = n^3$
\end{itemize}

M.T. 1. eset $\rightarrow S(n) \in \Theta(n^3)$.

Sajnos az elemi műveletek száma nem javult, nem nyertünk semmit.

Volker Strassen észrevétele: 7 db $\frac{n}{2} \times \frac{n}{2}$-es szorzással is megkaphatjuk az eredményt.
\begin{flalign*}
    P_1 &= A_{11} \cdot (B_{12} - B_{22}) &&\\
    P_2 &= (A_{11} + A_{12}) \cdot B_{22} &&\\
    P_3 &= (A_{21} + B_{22}) \cdot B_{11} &&\\
    P_4 &= A_{22} \cdot (B_{21} - B_{11}) &&\\
    P_5 &= (A_{11} + A_{22}) \cdot (B_{11} + B_{22}) &&\\
    P_5 &= (A_{12} - A_{22}) \cdot (B_{21} + B_{22}) &&\\
    P_5 &= (A_{11} - A_{21}) \cdot (B_{11} + B_{12})
\end{flalign*}
\begin{flalign*}
    C_{11} &= P_5 + P_4 - P_2 + P_6 &&\\
    C_{12} &= P_1 + P_2 &&\\
    C_{21} &= P_3 + P_4 &&\\
    C_{22} &= P_5 + P_1 - P_3 - P_7
\end{flalign*}

Összesen:
\begin{itemize}
    \item 7 db szorzás (7 db rekurzív hívás $\frac{n}{2} \times \frac{n}{2}$ méretű mátrixokra)
    \item 18 db összeadás
\end{itemize}

$
M(n) =
\begin{cases}
    1 & n=1\\
    7M\left(\frac{n}{2}\right) & n \geq 2\\
\end{cases}
$
 és 
$
S(n) =
\begin{cases}
    0 & n=1\\
    7S\left(\frac{n}{2}\right) + 18\left(\frac{n}{2}^2\right) & n \geq 2\\
\end{cases}
$

M.T. ezekre a rekurzív függvényekre:
\begin{itemize}
    \item $M(n) = 7M\left(\frac{n}{2}\right)$
    \item $a = 7$, $b = 2$, $f(n) = 0$
\end{itemize}

M.T. 1. eset: mivel $f(n) \in \mathcal{O}(n^{log_2 7} - \mathcal{E})$ valamilyen $\mathcal{E} > 0$-ra (pl. $\mathcal{E}=0.5$), ezért $M(n) \in \Theta(n^{log_2 7})$.


\begin{itemize}
    \item $S(n) = 7S\left(\frac{n}{2}\right) + \frac{9}{2}n^2$
    \item $a = 7$, $b = 2$, $f(n) = 4.5n^2$
    \item $n^{\log_2 7} = n^{2.81}$
\end{itemize}

M.T. 1. eset: mivel $f(n) \in \mathcal{O}(n^{log_2 7} - \mathcal{E})$ pl. $\mathcal{E}=0.1$, ezért $S(n) \in \Theta(n^{log_2 7})$.
\begin{flalign*}
    C_{12}
    &= P_1 + P_2 &&\\
    &= A_{11}(B_{12} - B_{22}) + (A_{11} + A_{12}) B_{22} &&\\
    &= A_{11}B_{12} - \text{\st{$A_{11}B_{22}$}} + \text{\st{$A_{11}B_{22}$}} + A_{11}B_{22}
\end{flalign*}
$C_{21}$ hasonlóan
\begin{flalign*}
    C_{11} &= P_5 + P_4 - P_2 + P_6 &&\\
    &= (A_{11} - A_{22})(B_{11} + B_{22}) + A_{21}(B_{21}-B_{11})-(A_{11} + A_{12})B_{22} + (A_{12}-A_{22})(B_{21}+B_{22}) &&\\
    &= \cdots
\end{flalign*}

\textbf{HF.} kérdés: adott $A, B, C \in \mathcal{R}^{n \times n}$ mátrixok esetén a számolások elvégzése nélkül eldönthető-e, hogy AB = C?

\clearpage
\subsection{Randomizált algoritmusok}
\begin{itemize}
    \item Las Vegas típusú: sosem téved, de néha lassú, pl. quicksort
    \item Monte Carlo típusú: gyors, de néha téved, pl. Freivalds algoritmus
\end{itemize}

\subsubsection*{Freivalds algoritmus}
\begin{itemize}
    \item Véletlenszerűen választunk egy $\alpha \in \{0, 1\}^n$ bitvektort és kiszámoljuk ($\Theta(n^2)$ költséggel) a $\beta = (AB) \cdot \alpha = A \cdot (B \alpha)$ és $\gamma = C \cdot \alpha$ vektorokat.
    \item Ha $\beta \neq \gamma$, akkor visszatérünk HAMIS válasszal (itt biztosan nem tévedtünk).
    \item Ha $\beta = \gamma$, akkor visszatérünk IGAZ válasszal (lehet, hogy tévedtünk).
\end{itemize}

\textbf{Észrevétel:} az utóbbi eset az n dimenziós $\alpha$ bitvektorok legfeljebb felére teljesülhet.
Ha $\alpha$-t véletlenszerűen választjuk, akkor a tévedés valószínűsége $\leq \frac{1}{2}$.
Az algoritmus: végezzük el ezt 100-szor egymás után $\rightarrow$ a tévedés valószínűsége $\leq (\frac{1}{2})^{100}$.

\textbf{Észrevétel bizonyítása:}\\
Ha $AB \neq C$ és valamilyen $\alpha$-ra $AB \alpha = C \alpha$, akkor
\begin{itemize}
    \item $AB-C$ mátrix nem az azonosan 0 mátrix.
    \item tfh. az $i$. sor $j$. eleme $d > 0$.
    \item legyen $\alpha'$ az a vektor, ha $\alpha$-ban a $j$. elemet átváltjuk (bit flip).
\end{itemize}
Ekkor $(AB-C) \alpha'$ $i$. koordinátája megváltozott ($-d$-re, $+d$-re), tehát $(AB-C) \alpha' \neq 0$.
Minden $\alpha$ vektorhoz találhatunk $\alpha'$ vektort, ami kimutatja az $AB \neq C$ tényt, és különböző $\alpha$ vektorokhoz különböző $\alpha'$-t kapunk $\rightarrow$ a hamis eredményt adó vektorok száma $\geq$ helyes eredményt adó vektorok száma.





