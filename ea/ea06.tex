\section{előadás (2025. október 14.)}
\subsection{Gyors Fourier transzformált}
Legyen $C(x) = c_0 + c_1 x + \cdots + c_{2n-1} x^{2n-1}$.
Tfh. ismerjük $C(x)$ helyettesítési értékeit az $\omega_{1, 2n}, \omega_{2, 2n}, \cdots, \omega_{2n, 2n}$ $2n$-edik komplex egységgyökökön.\\
Kérdés: $c_0, c_1, \cdots, c_{2n-1}$.

Bevezetünk egy $D(x) = d_0 + d_1 x + \cdots + d_{2n-1} x^{2n-1}$ polinomot, ahol $d_S = C(\omega_{S,2n})$ $(S = 0, 1, \cdots, 2n-1)$ (semmi pánik, $\omega_{0, 2n} = \omega_{2n, 2n}$).

\textbf{Állítás:} $D(\omega_{j, 2n}) = 2n c_{2n-j}$ vagyis $c_{2n-j} = \frac{1}{2n} D(\omega_{j, 2n})$ $(j = 1, 2, \cdots, 2n)$.

\textbf{Bizonyítás:}\\
\begin{flalign*}
    D(\omega_{j, 2n}) &= \overset{2n-1}{\underset{S=0}{\sum}} d_S \cdot \omega_{j, 2n}^{S} &&\\
    &= \overset{2n-1}{\underset{S=0}{\sum}} C(\omega_{S, 2n}) \cdot \omega_{j, 2n}^{S} &&\\
    &= \overset{2n-1}{\underset{S=0}{\sum}} \left( \overset{2n-1}{\underset{S=0}{\sum}} c_t \cdot \omega_{S, 2n}^{t} \right) \omega_{j, 2n}^{S}
\end{flalign*}

Kettős összegzés
\begin{figure}[H]
  \includegraphics[width=10cm]{ea/img/ea06_double_sum}
\end{figure}
"a sorrend felcserélhető"

Kitérő (C. F. Gauss):
\begin{flalign*}
    1 + 2 + \cdots + 49 + 50 = \frac{50 \cdot 51}{2} &&\\
    \underbrace{\frac{50}{51} + \frac{49}{51} + \cdots + \frac{2}{51} + \frac{1}{51}}_{50 \cdot 51}
\end{flalign*}

Visszatérve az eredeti gondolatmenethez:\\
$\overset{2n-1}{\underset{S=0}{\sum}} \left( \overset{2n-1}{\underset{t=0}{\sum}} c_t + \omega_{S, 2n}^t \right) \omega_{j, 2n}^{S} = \overset{2n-1}{\underset{t=0}{\sum}} c_t \text{\fbox{$\overset{2n-1}{\underset{S=0}{\sum}} \omega_{S, 2n}^t \cdot \omega_{j, 2n}^S$}} \text{ ez könnyen számolható:}$

\begin{flalign*}
    \omega_{S, 2n}^t \cdot \omega_{j, 2n}^S &= (e^{2 \pi i \cdot \frac{S}{2n}})^t (e^{2 \pi i \cdot \frac{j}{2n}})^S &&\\
    &= e^{2 \pi i \cdot \frac{S t}{2n}} \cdot e^{2 \pi i \cdot \frac{S j}{2n}} &&\\
    &= e^{2 \pi i \cdot \frac{S t}{2n} + 2 \pi i \cdot \frac{S j}{2n}} &&\\
    &= e^{\frac{(2 \pi i \cdot S t + 2 \pi i \cdot S j)}{2n}} &&\\
    &= e^{\frac{S \cdot 2 \pi i (t + j)}{2n}} &&\\
    &= \left(e^{\frac{2 \pi i (t + j)}{2n}}\right)^S &&\\
    &= \omega_{t+j, 2n}^S
\end{flalign*}

Az egész így:\\
$\overset{2n-1}{\underset{t=0}{\sum}} c_t \text{\fbox{$\overset{2n-1}{\underset{S=0}{\sum}} \omega_{t+j, 2n}^S$}}$

Egy pillanatra bevezetve az $\omega = \omega_{t+j, 2n}$ jelölést $\overset{2n-1}{\underset{S=0}{\sum}} \omega^S = 1 + \omega + \omega^2 + \cdots + \omega^{2n-1}$.

Vegyük észre (bizonyítsuk teljes indukcióval): $(\omega - 1)(1 + \omega + \omega^2 + \cdots + \omega^{2n-1}) = \omega^{2n} - 1$.

Mivel $\omega = \omega_{t+j, 2n}$, egy $2n$-edik komplex egységgyök, ezért $\omega^{2n} - 1 = 0$.\\
Ebből következik, hogy ha $\omega \neq 1$, akkor $1 + \omega + \omega^2 + \cdots + \omega^{2n-1} = 0$.\\
Ha $\omega = 1$, akkor $1 + \omega + \omega^2 + \cdots + \omega^{2n+1} = 1 + 1 + \cdots + 1 = 2n$.\\
Mikor lesz $\omega_{t+j, 2n} = 1$? Ha $t = 2n-j$!\\
Valóban, $t + j$ többszöröse kell, hogy legyen $2n$-nek, azonban $1 \leq j \leq 2n$ és $0 \leq t \leq 2n-1$ miatt ez csak $t+j=2n$ esetén áll fenn.

Így $\overset{2n-1}{\underset{t=0}{\sum}} c_t \underbrace{\overset{2n-1}{\underset{S=0}{\sum}} \omega_{t+j, 2n}^S}_{\text{csak $t=2n-j$ esetén $\neq 0$}} = c_{2n-j} \cdot 2n$.

\subsubsection*{Nagy egészek szorzása}
$A$, $B$ n db számjegyből áll (tízes számrendszerben felírva).\\
Kérdés: $A \cdot B = ?$

\textbf{Általános iskola}\\
\ul{234} \cdot 425\\
936\\
\- 468\\
\- 1170\\
$\overline{99450}$\\
$\mathcal{O}(n^2)$ "elemi" szorzás

\textbf{Oszd meg és uralkodj trükk} (magunk is kitalálhatjuk):

$A = A_1 \cdot 10^{\frac{n}{2}} + A_0$\\
$B = B_1 \cdot 10^{\frac{n}{2}} + B_0$\\
$A_1, A_0, B_1, B_0$ $\frac{n}{2}$ db számjegyből áll.

$AB = (A_1 \cdot 10^{\frac{n}{2}} + A_0) (B_1 \cdot 10^{\frac{n}{2}} + B_0) = A_1 B_1 \cdot 10^n + (A_1 B_0 + A_0 B_1) \cdot 10^{\frac{n}{2}} + A_0 B_0$

\textbf{Nyertünk valamit azzal}, hogy az eredeti feladatot négy feleakkora méretű feladatra redukáltuk?\\
$T(n) = 4 \cdot T(\frac{n}{2}) \rightarrow T(n) = \Theta(n^{\log_2 4}) = \Theta(n^2)$ :( \\
Azért ilyen olcsón nem várhatunk sokat!

Viszont $A_0 B_0, A_0 B_1, A_1 B_0, A_1 B_1$ helyett igazából $A_0 B_0, A_0 B_1 + A_1 B_0, A_1 B_1$ számolandó.\\
Piszkos trükk:
$(A_1 + A_0)(B_1 + B_0)$ egy szorzás.\\
Eredmény: $A_1 B_1 + \text{\fbox{$A_0 B_1 + A_1 B_0$}} + A_0 B_0$.\\
Így $A_0 B_1 + A_1 B_0 = (A_1 + A_0)(B_1 B_0) - A_1 B_1 - A_0 B_0$

Így a feladat visszavezethető csupán 3 feleakkora méretű feladatra.\\
$T(n) = 3T \left(\frac{n}{2} \right) \rightarrow T(n) = \Theta(n^{\log_2 3})$ :)

\subsection{Dinamikus programozás}
Alapelv: itt is ugyanaz, mind az oszd meg és uralkodj algoritmusoknál.
A feladatot (rekurzívan) visszavezetjük hasonló, csak kisebb méretű feladatok megoldására.

Egy tipikus oszd meg és uralkodj algoritmusnál (összefésüléses rendezés):
\begin{figure}[H]
  \includegraphics[width=10cm]{ea/img/ea06_merge_sort}
\end{figure}
Független részproblémák

Nem mindig ez a helyzet!\\
Illusztráció:\\
\begin{tikzpicture}
    \node(node1_1) at (0,0) {$\binom{n}{k}$};
    \node(node2_1) at (-1,-1) {$\binom{n-1}{k-1}$};
    \node(node2_2) at (1,-1) {$\binom{n-1}{k}$};
    \node(node3_1) at (-1.5,-2) {$\binom{n-2}{k-2}$};
    \node(node3_2) at (-0.5,-2) {$\binom{n-2}{k-1}$};
    \node(node3_3) at (0.5,-2) {$\binom{n-2}{k-1}$};
    \node(node3_4) at (1.5,-2) {$\binom{n-2}{k}$};
    \draw[-Latex] (node3_1.north) -- (node2_1.south);
    \draw[-Latex] (node3_2.north) -- (node2_1.south);
    \draw[-Latex] (node3_3.north) -- (node2_2.south);
    \draw[-Latex] (node3_4.north) -- (node2_2.south);
    \draw[-Latex] (node2_1.north) -- (node1_1.south);
    \draw[-Latex] (node2_2.north) -- (node1_1.south);
\end{tikzpicture}\\
"Átfedő részproblémák"

\textbf{Összefoglalva}
\begin{itemize}
    \item \textbf{Oszd meg és uralkodj} tipikusan fentről lefelé.
    \item \textbf{Dinamikus programozás} tipikusan lentről felfelé és a részproblémák megoldását egy táblázatban jegyezzük fel (létezik fentről lefelé is $\rightarrow$ memoizálás).
\end{itemize}

\textbf{Mi mindenre jó?}\\
Elhelyezünk egy sorban különböző címletű érméket, és két játékos felváltva elvehet egy érmét a sor valamelyik végéről.

\begin{tikzpicture}[
    roundnode/.style={circle, draw=black, minimum size=0.6cm, inner sep=0.5mm}
]
\node[roundnode](c1){$5$};
\node[roundnode](c2)[right = 0.2cm of c1]{$100$};
\node[roundnode](c3)[right = 0.2cm of c2]{$5$};
\node[roundnode](c4)[right = 0.2cm of c3]{$20$};
\node[roundnode](c5)[right = 0.2cm of c4]{$50$};
\node[roundnode](c6)[right = 0.2cm of c5]{$200$};
\node[roundnode](c7)[right = 0.2cm of c6]{$100$};
\node[roundnode](c8)[right = 0.2cm of c7]{$50$};
\node[roundnode](c9)[right = 0.2cm of c8]{$10$};
\end{tikzpicture}

Milyen stratégiát kövessen Aliz az elvett érmék összértékének maximalizálására, feltéve, hogy Robi is ugyanezt akarja maga számára?