\section{előadás (2025. november 11.)}
\subsection{Optimális szekvenciaillesztés}
(Levenshtein távolság)

Pozíciók:
\begin{flalign*}
    &X = (x_1, x_2, \cdots, x_m) \rightsquigarrow \{1, 2, \cdots, m\} &&\\
    &Y = (y_1, y_2, \cdots, y_n) \rightsquigarrow \{1, 2, \cdots, n\} &&
\end{flalign*}

Egy olyan $M$ párosítást (pontosabban szekvenciaillesztést) keresünk $\{1, 2, \cdots, m\}$ és $\{1, 2, \cdots, n\}$ között, melyre a következő mennyiség minimális:
\begin{itemize}
    \item egyrészt annyiszor egy $g>0$ valós számot, ahány pozíció nem szerepel $M$-ben
    \item másrészt ha $(i, j) \subseteq M$, akkor egy $c[x_i, y_j] \geq 0$ valós számot
\end{itemize}

\subsubsection*{DP algoritmus}
\textbf{1. Részproblémák}\\
$R[i, j] \rightarrow$ optimális szekvenciaillesztés meghatározása az $X_i = (x_1, x_2, \cdots, x_i)$ és $Y_j = (y_1, y_2, \cdots, y_j)$ prefixek között.

\textbf{2. Optimális részstruktúra tulajdonság}\\
\textbf{Észrevétel}\\
Legyen $M$ szekvenciaillesztés (akármilyen) $X_i$ és $Y_j$ között.
\setulcolor{black}
Ekkor három eset lehet \ul{csak}:
\begin{enumerate}
    \item $(i, j) \in M$
    \item $i$ nem szerepel $M$-ben első tagként
    \item $j$ nem szerepel $M$-ben második tagként
\end{enumerate}

Ami hiányzik: $j$ szerepel második tagként, $i$ nem szerepel első tagként.
$(i, j) \notin M$.\\
Ám ekkor\\
\begin{tikzpicture}
\node(t1){$1$};
\node(t2)[right = 0.4cm of t1]{$2$};
\node(t3)[right = 0.4cm of t2]{$3$};
\node(b1)[below = 0.4cm of t1]{$1$};
\node(b2)[right = 0.4cm of b1]{$2$};
\node(b3)[right = 0.4cm of b2]{$3$};

\node(cdots)[right = 0.5cm of b3]{$\cdots$};
\node(ti)[right = 1cm of t3]{$i$};
\node(bj)[right = 0.5cm of cdots]{$j$};

\draw (t3) -- (bj);
\draw (b3) -- (ti);

\node[above right = 0cm and 0.7cm of bj]{$M$};
\end{tikzpicture}\\
ami nem szekvenciaillesztés.

\textbf{Állítás}\\
Legyen $M$ \ul{optimális} szekvenciaillesztés $X_i$ és $Y_j$ között.\\
Ekkor
\begin{enumerate}
    \item Ha $(i, j) \in M$, akkor $M\setminus\{(i, j)\}$ optimális szekvenciaillesztés $X_{i-1}$ és $Y_{j-1}$ között.
    \item Ha $i$ nem szerepel $M$-ben első tagként, akkor $M$ optimális szekvenciaillesztés $X_{i-1}$ és $Y_j$ között.
    \item Ha $j$ nem szerepel $M$-ben második tagként, akkor $M$ optimális szekvenciaillesztés $X_i$ és $Y_{j-1}$ között.
\end{enumerate}

\textbf{Bizonyítás}\\
\begin{enumerate}
    \item Ha $X_{i-1}$ és $Y_{j-1}$ között lenne egy $M \setminus \{(i, j)\}$-nél kedvezőbb $M^*$ szekvenciaillesztés, akkor $M^* \cup \{(i, j)\}$ egy $M$-nél kedvezőbb szekvenciaillesztés lenne $X_i$ és $Y_j$ között, \textbf{ellentmondás}.
    \item Ha lenne $X_{i-1}$ és $Y_{j}$ között egy egy $M$-nél kedvezőbb $M^*$ szekvenciaillesztés, akkor itt lenne $X_i$ és $Y_j$ között is.
    \item Analóg (2.)-hoz.
\end{enumerate}

\textbf{3. Rekurzió}\\
Jelölje $l[i,j]$ az $R[i,j]$ megoldásának "értékét".

\textbf{"Alapeset"}\\
Az egyik (vagy mindkettő) karaktersorozat üres
\begin{flalign*}
    &l[i,0] = i - g &&\\
    &l[0,j] = j + g &&
\end{flalign*}
($M = \emptyset$ az egyetlen párosítás)

\textbf{"Általában"}\\
\begin{flalign*}
    &(A) \rightarrow l[i,j] = l[i-1, j-1] + c[x_i, y_j] &&\\
    &(B) \rightarrow l[i,j] = l[i-1, j] + g &&\\
    &(C) \rightarrow l[i,j] = l[i, j-1] + g &&
\end{flalign*}

Sajnos nem tudjuk, hogy $(A)$ és $(B)$ és $(C)$ közül melyik teljesül, így mindhármat megvizsgáljuk és a legkedvezőbbet választjuk.

$
l[i, j] = min
\begin{cases}
    l[i-1, j-1] + c[x_i, y_j]\\
    l[i-1, j] + g\\
    l[i, j-1] + g
\end{cases}
$

[szakirodalomban sok helyen (bizonyítás nélkül) az áll, hogy ha $x_i = y_j$, akkor $l[i, j] = l[i-1, j-1]$ (feltéve, hogy $c["p", "q"] \geq 0$ mindig)]


\textbf{4. Optimális megoldás értéke}\\
Az $(m+1) \times (n+1)$-es $l[i,j]$ táblázatot töltjük ki fentről lefelé, soronként balról jobbra.

\begin{tabular}{l l}
    $l[i, j] =$ & \raisebox{-.5\height}{\includegraphics[width=6cm]{ea/img/ea09_matrix}} \\
\end{tabular}

Az első sorral kezdünk, az aktuális érték a három szomszédból számítódik (amiket már ismerünk) 

Számítási bonyolultság: $\mathcal{O}(mn)$ cella, cellánként $\mathcal{O}(1)$ számítás $\rightarrow \mathcal{O}(mn)$.

\textbf{5. Optimális megoldás}\\
$\rightarrow$ optimális $M$

$l[i,j]$ számításakor az $(A)$, $(B)$ és $(C)$ közül a legkedvezőbbet adót magát is feljegyezzük.

\angledarrow{0}
\angledarrow{-45}
\angledarrow{-90}
(vizuálisan)

Ezeket tároljuk egy $k[i,j]$ táblázatban, amelynek elkészülte után a jobb alsó sarokból a nyilak mentén a "margóig" lépkedünk.

Ha $k[i,j] = $ \angledarrow{135}, akkor $M = M \cup \{(i, j)\}$ az üres $M$-ből indulva.


\subsection{Leghosszabb közös részsorozat}
\begin{flalign*}
    &X = (x_1, x_2, \cdots, x_m) &&\\
    &Y = (y_1, y_2, \cdots, y_n) &&
\end{flalign*}
Egy leghosszabb olyan $Z = (z_1, z_2, \cdots, z_k)$-t keresünk, amely megkapható $X$-ből is és $Y$-ból is "néhány" karakter törlésével.

Hasonló DP algoritmus működik itt is, mint az előbb.

\fbox{\textbf{1.}} $R[i,j]$ $X_i$ és $Y_j$ egy leghosszabb közös részsorozatának meghatározása\\
\fbox{\textbf{2.}} \textbf{Állítás}\\
Legyen $Z_k = (z_1, z_2, \cdots, z_k)$ az $X_m = (x_1, x_2, \cdots, x_m)$ és $Y_n = (y_1, y_2, \cdots, y_n)$ egy leghosszabb közös részsorozata (LKR).

\begin{enumerate}
    \item Ha $x_m = y_n$, akkor $z_k = x_m = y_n$ ÉS $Z_{k-1}$ egy LKR $X_{m-1}$-hez és $Y_{n-1}$-hez.
    \item Ha $x_m \neq z_k$, akkor $Z_{k}$ egy LKR $X_{m-1}$-hez és $Y_n$-hez.
    \item Ha $y_n \neq z_k$, akkor $Z_{k}$ egy LKR $X_m$-hez és $Y_{n-1}$-hez.
\end{enumerate}

\fbox{\textbf{3.}}
Legyen $l[i, j]$ az $R[i,j]$ optimális megoldásának értéke (hossza).
\begin{flalign*}
    &(A) \rightarrow l[i,j] = l[i-1, j-1] + 1 &&\\
    &(B) \rightarrow l[i,j] = l[i-1, j] &&\\
    &(C) \rightarrow l[i,j] = l[i, j-1] &&
\end{flalign*}

$
l[i, j] = min
\begin{cases}
    0 & \text{ha $i=0$ vagy $j=0$ (rekurzió "elvarrása")}\\
    l[i-1, j-1] + 1 & \text{ha $i,j \geq 1$ és $x_i = y_j$ $(A)$}\\
    min \{l[i-1, j], l[i, j-1]\} & \text{ha $i,j \geq 1$ és $x_i \neq y_j$ $(B)$, $(C)$}\\
\end{cases}
$