\section{előadás (2025. november 25.)}
\subsubsection*{Hátizsák feladat}
Legyenek $t_1, t_2, \cdots, t_n$ tárgyak.
Minden $t_i$ tárgynak van egy $w_i$ súlya és egy $v_i$ értéke.
Adott egy hátizsák $W$ kapacitással.
Válasszunk ki a tárgyak közül néhányat úgy, hogy a kiválasztott tárgyak összsúlya ne haladja meg a hátizsák $W$ kapacitását és az összértéke a lehető legnagyobb.

Ismert, hogy ez egy NP-nehéz feladat.
Ezzel együtt van egy különös tulajdonsága: bármely $\mathcal{E} > 0$ valós számhoz van olyan polinomiális algoritmus, amely által szolgáltatott megoldásban az összérték legalább $\frac{1}{1 + \mathcal{E}}$-szorosa az optimális összértéknek.
(aprópénzre váltva legalább $99\%$-a, legalább $99,9\%$-a, legalább $99,99\%$-a, és így tovább)

Hátizsák feladat input:
$w_1, v_1, w_2, v_2, \cdots, w_n, v_n, W$ mind pozitív egészek.

\textbf{DP algoritmus (általánosan nem polinomiális!)}\\
\textbf{1. Részproblémák}\\
$R[i,j] \rightarrow$ tárgyak $t_1, t_2, \cdots, t_i$, hátizsák kapacitás: $j$

\textbf{2. Optimális részstruktúra tulajdonság}\\
Legyen Opt egy optimális (maximális összértékű) megoldása $R[i,j]$-nek.

Most $t_i \notin$ Opt  (Opt most egy tárgyhalmaz) esetén Opt optimális megoldása $R[i-1,j]$-nek.
Valóban, ha lenne $R[i-1,j]$-nek Opt-nál nagyobb összértékű megoldása, az Opt-nál nagyobb összértékű megoldása lenne $R[i,j]$-nek is, \textbf{ellentmondás}.

Másrészt $t_i \in Opt$ esetén Opt$\setminus \left\{t_i\right\}$ egy maximális összértékű megoldása $R[i-1,j-w_i]$-nek.
Valóban, ha lenne $R[i-1, j-w_i]$-nek Opt$\setminus \left\{t_i\right\}$-nél nagyobb összértékű megoldása, akkor ezt kiegészítve $t_i$-vel, $R[i,j]$-nek egy Opt-nál nagyobb összértékű megoldását kapnánk, \textbf{ellentmondás}.

\textbf{3. Rekurzió}\\
Jelölje $h[i,j]$ az $R[i,j]$ optimális megoldásánál az összértéket.

Nyilván $h[i,0]$ és $h[0,j]$ nulla.

Különben, vagyik amikor $i, j \geq 1$, akkor 

\fbox{\textbf{A}} ha $w_i > j$, akkor $t_i$ nem lehet benne az optimális megoldásban, így $h[i,j]=h[i-1,j]$.

\fbox{\textbf{B}} ha viszont $w_i \leq j$, akkor $t_i$ vagy benne van az optimális megoldásban, vagy nincs, és nem is tudjuk melyik áll fenn a kettő közül.
Így szokásos módon mindkét lehetőséget megvizsgáljuk, és a kedvezőbb lesz a befutó:

$
h[i, j] = max
\begin{cases}
    v_i + h[i-1, j-w_i] & \text{(benne van)}\\
    h[i-1, j] & \text{(nincs benne)}\\
\end{cases}
$

\textbf{4. Táblázat}\\
Az $(n+1) \times (w+1)$-dimenziós $h[i,j]$ táblázatot töltjük ki (felülről lefelé)

\begin{tabular}{l l}
    $h[i, j] =$ & \raisebox{-.5\height}{\includegraphics[width=8cm]{ea/img/ea11_matrix1}} \\
\end{tabular}

Oszlopok: $j = 0, 1, \cdots, W$

\ul{Költség:}\\
egy cella: $\mathcal{O}(1)$\\
cellák száma: $\mathcal{O}(nW)$\\
összesen: $\mathcal{O}(nW)$, nem polinomiális!

A táblázatban az oszlopok száma $W+1$, ez $W$, mint input input méretében (ami $\log W$) exponenciálisan nagy.

Megjegyzés: kis súlyok esetén az algoritmus polinomiális (pontosítva: $w_1 + w_2 + \cdots + w_n \leq W$ esetén a feladat triviális, itt nem ilyenekről van szó)

\textbf{5. Optimális tárgyhalmaz}\\
$h[i,j]$ számolásakor feljegyezzük, hogy $t_i$ beválasztása vagy kihagyása volt kedvezőbb, a jobb alsó sarkoból indulva ezek segítségével rekonstruálható egy optimális megoldás.

\textbf{Lássunk ezek után egy alternatív megközelítést}\\
Hátizsák feladat kicsit másképpen
\begin{itemize}
    \item input változatlan
    \item "Célfüggvény" legalább V összértéket szeretnénk elérni az ehhez szükséges tárgyak minimális összsúlya mellett.
\end{itemize}

DP itt is működik (ezt talán mindenki érzi)

\textbf{1. Részproblémák}\\
$R[i,j]$ a $t_1, t_2, \cdots, t_i$ tárgyakból válaszhatunk és legalább $j$ összértéket akarunk elérni minimális összsúly mellett.

Jelölje $h^*[i,j]$ a minimális összsúlyt.

\fbox{\textbf{2.}}-\fbox{\textbf{3.}} nagyon hasonlít az előbbiekhez.

\textbf{4. Táblázat}\\
$h^*[i, j] =$
\raisebox{-.5\height}{
\begin{tikzpicture}
\node(img) {\includegraphics[width=6cm]{ea/img/ea11_matrix2}};
\node[above right=-0.5cm and -4.5cm of img]{$v_1 + v_2 + \cdots + v_i$};
\node[above right=-0.5cm and -1.3cm of img]{$v_1 + v_2 + \cdots + v_n$};

\node(arr1end) at (0.1, -0.5){};
\node(arr2end) at (2.2, -2.1){};

\node(wi) at (2.1, 0.2){$w_1 + w_2 + \cdots + w_i$};
\node(wn) at (3, -2.8){$w_1 + w_2 + \cdots + w_n$};

\draw[-Latex] (wi.south) -- (arr1end.center);
\draw[-Latex] (wn.north) -- (arr2end.center);
\end{tikzpicture}
}


Csak az "alsó" részt töltjük ki.

Hogyan használható ez a táblázat az "eredeti" hátizsák feladat megoldására (ahol $W$ a hátizsák kapacitása)?
Az utolsó sorban balról jobbra haladva megkeressük az utolsó $\leq W$ értéket, ennek oszlopindexe lesz az elérhető maximális összsúly.

\ul{Költség:}\\
egy cella: $\mathcal{O}(1)$\\
cellák száma: legyen $v^* = max v_i$\\
sorok száma: $\mathcal{O}(n)$\\
oszlopok száma: $\mathcal{O}(n v^*)$\\
összesen: $\mathcal{O}(n^2 v^*)$, nem polinomiális!

Azonban kis értékek esetén polinomiális!
Pl. ha $v^*=(n^3)$
