\section{előadás (2025. december 2.)}
\subsubsection*{Közelítő hátizsák feladat}
\textbf{Tétel}\\
Minden $\mathcal{E} > 0$ valós számhoz létezik olyan polinomiális algoritmus, amely által visszaadott összérték a hátizsák problémára legalább $\frac{1}{1 + \mathcal{E}}$-szorosa az optimális megoldásnak.

\textbf{Hátizsák feladatunk}
\begin{flalign*}
    &t_1 \rightarrow (w_1, v_1) &&\\
    &t_2 \rightarrow (w_2, v_2) &&\\
    &\vdots &&\\
    &t_n \rightarrow (w_n, v_n) &&\\
    &W                          &&
\end{flalign*}
Nyilván azok a tárgyak, amelyek súlya nagyobb, mint W nem lehetnek benne egyetlen megoldásban sem, ezért feltehetjük, hogy $w_i \leq W (1 \leq i \leq n)$

Nyilván elég a tételt olyan $\mathcal{E}$ pozitív számokra igazolni, amelyek pozitív egészek reciprokai.

Definiálhatunk két másik hátizsák feladatot.
Legyen $b = \frac{\mathcal{E}}{2n} v^*$ (ahol $v^* = \underset{i}{max} v_i$)

A súlyok és a hátizsák kapacitása változatlan.

\fbox{\textbf{I.}} az értékek $\tilde{v}_i = \left\lceil \frac{v_i}{b} \right\rceil b$\\
\fbox{\textbf{II.}} az értékek $\hat{v}_i = \left\lceil \frac{v_i}{b} \right\rceil = \frac{\tilde{v}_i}{b}$

Jegyezzük meg, hogy $v_i \leq \tilde{v}_i \leq v_i + b$.

Nyilván \fbox{\textbf{I.}} és \fbox{\textbf{II.}} optimális megoldása ugyanaz $\rightarrow \mathcal{J}$

Oldjuk meg \fbox{\textbf{II.}}-t a múlt végén ismertetett DP algoritmussal.

\textbf{Mi is volt ez a DP algoritmus?}

A részproblémák úgy néztek ki, hogy a $t_1, t_2, \cdots, t_i$ tárgyakkal akartunk legalább $j$ értéket elérni minimális összsúly mellett $\rightarrow h^*(i,j)$ volt a min összsúly.

[$1 \leq i \leq n$, $0 \leq j \leq v_1 + v_2 + \cdots + v_i$]

Ekkor:
\begin{flalign*}
&h^*[i,j] =
\begin{cases}
    0 & \text{ha $j = 0$}\\
    w_i + h^*[i-1, max \left\{ 0, j - v_i \right\}] & \text{ha $j > v_1  + v_2 + \cdots + v_{i-1}$}\\
    min
    \begin{cases}
        w_i + h^*[i-1, max \left\{ 0, j - v_i \right\}] & \\
        h^*[i-1, j] &
    \end{cases} & \text{ha $1 \leq j \leq v_1  + v_2 + \cdots + v_{i-1}$}
\end{cases}
&&\\
\end{flalign*}

\textbf{Táblázat}\\
\begin{tikzpicture}
\node(img) {\includegraphics[width=6cm]{ea/img/ea12_matrix.pdf}};
\node[above right=-0.3cm and -3.5cm of img]{opt. mo, ez az oszlopindex};
\node[below right=-0.2cm and -4.9cm of img]{$\rightarrow$ utolsó $\leq W$ érték};

\draw[-Latex] (0.52, -2.43) -- (0.52, 2.7);
\end{tikzpicture}

Költség: n sor, $\leq n v^*$ oszlop, cellánként $\mathcal{O}(1)$ számolás $\rightarrow \mathcal{O}(n^2 v^*)$

\fbox{\textbf{II.}} megoldásának költsége\\
$\mathcal{O}(n^2 \underset{i}{max} \hat{v}_i)$

Mennyi ez?
Legyen $\underset{i}{max} v_i = v_j$.
Ekkor $\underset{i}{max} \hat{v}_i = \hat{v}_j$.

Igen ám, de
$\hat{v}_j =
\left\lceil \frac{v_j}{b} \right\rceil =
\left\lceil \frac{v^*}{b} \right\rceil =
\left\lceil \frac{v^*}{\frac{\mathcal{E}}{2n} v^*} \right\rceil = 
\left\lceil \frac{1}{\frac{\mathcal{E}}{2n}} \right\rceil = 
\left\lceil \frac{2n}{\mathcal{E}} \right\rceil = 
\frac{2n}{\mathcal{E}}
$
($\frac{1}{\mathcal{E}}$ pozitív egész, $2n$ is)

Visszahelyettesítve $\mathcal{O}(n^2 \underset{i}{max} \hat{v}_i)$-ba ez $\mathcal{O}(n^3 \mathcal{E}^{-1})$ ami rögzített $\mathcal{E}$ mellett polinomiális!

Hátra van még az "$\frac{1}{1 + \mathcal{E}}$ rész".

Idézzük fel, hogy $\mathcal{J}$ nem csak a $\hat{v}_i$ értékekkel, de a $\tilde{v}_i$ értékekkel is optimális megoldás.

Mit mondhatunk $\mathcal{J}$-ről a $v_i$ értékek esetén (eredeti hátizsák feladat)?

Tekintsük az eredeti feladat egy tetszőleges $\mathcal{J}^*$ megengedett megoldását (ebben benne van az optimális is).

Megmutatjuk, hogy $\frac{1}{1 + \mathcal{E}} \underset{i \in \mathcal{J}^*}{\sum} v_i \leq \underset{i \in \mathcal{J}}{\sum} v_i$.
Vagy kicsit átrendezve: $\underset{i \in \mathcal{J}^*}{\sum} v_i \leq (1 + \mathcal{E}) \underset{i \in \mathcal{J}}{\sum} v_i$.
(Persze $\mathcal{J}$ is egy megengedett megoldása az eredeti feladatnak, mert a súlyok és a kapacitás nem változik.)

Az optimális megoldást adó $\mathcal{J}^*$-gal ez pont a tétel állítása.

\textbf{"Bűvészkedünk"}\\
Idézzük fel, hogy $v_i \leq \tilde{v}_i \leq v_i + b$.
Most jön egy hosszú egyenlőtlenséglánc:\\
$\underset{i \in \mathcal{J}^*}{\sum} v_i \leq 
\underset{i \in \mathcal{J}^*}{\sum} \tilde{v}_i \leq 
\underset{i \in \mathcal{J}}{\sum} \tilde{v}_i \leq 
\underset{i \in \mathcal{J}}{\sum} (v_i + b) \leq 
\underset{i \in \mathcal{J}}{\sum} v_i + \underset{i \in \mathcal{J}}{\sum} b \leq 
\left(\underset{i \in \mathcal{J}}{\sum} v_i\right) + nb
$

$\mathcal{J}$ az optimális megoldás a $\tilde{v}_i$ értékekre.
Arra hajtunk, hogy $nb \leq \mathcal{E} \underset{i \in \mathcal{J}}{\sum} v_i$.

Piszkos trükk: nézzük az egyenlőtlenséglánc következő részét:
$
\underset{i \in \mathcal{J}}{\sum} \tilde{v}_i \leq 
\left(\underset{i \in \mathcal{J}}{\sum} v_i\right) + nb
$

Először is a $j$-edik tárgy (ennek a legnagyobb az értéke) önmaga megengedett megoldása \fbox{\textbf{I.}}-nél, így $\tilde{v}_j \leq \underset{i \in \mathcal{J}}{\sum} \tilde{v}_i$.
