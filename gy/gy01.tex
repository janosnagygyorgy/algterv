\section{gyakorlat (2025. szeptember 10.)}
\subsection{Stabil párosítás}
\subsubsection*{Naiv algoritmus (nem feltétlen termináló)}
Könyv: \url{../konyv.pdf#page=105}

\begin{tabular}{l l}
    $f_1 \rightarrow (l_2, l_1, l_3)$ & $l_1 \rightarrow (f_1, f_3, f_2)$ \\
    $f_2 \rightarrow \text{tetszőleges, pl. } (l_1, l_2, l_3)$ & $l_2 \rightarrow (f_3, f_1, f_2)$ \\
    $f_3 \rightarrow (l_1, l_2, l_3)$ & $l_3 \rightarrow \text{tetszőleges, pl. } (f_1, f_2, f_3)$ \\
\end{tabular}

\textbf{Algoritmus}
\begin{enumerate}
    \item Keressünk instabilitást a párosításban: egy fiúnak jobban tetszik egy másik lány ÉS a lánynak jobban tetszik egy másik fiú.
    \item Az (egyik) instabilitásbeli \ul{négyest} felcseréljük.
\end{enumerate}

\setulcolor{red}
\textbf{Kezdeti párosítás:} $M_0=\left\{\text{\ul{$(f_1, l_1)$}}, \text{\ul{$(f_2, l_2)$}}, (f_3, l_3)\right\}$, ez nem stabil.\\
\textbf{Instabilitás:} $\left[f_3-l_2\right]$ és \ul{$\left[f_1-l_2\right]$} (több van, ezért \ul{választunk} egyet)

\setulcolor{blue}
$M_1=\left\{\text{\ul{$(f_1, l_2)$}}, (f_2, l_1), \text{\ul{$(f_3, l_3)$}}\right\}$\\
\textbf{Instabilitás:} \ul{$\left[f_3-l_2\right]$} és $\left[f_3-l_1\right]$

\setulcolor{green}
$M_2=\left\{(f_1, l_3), \text{\ul{$(f_2, l_1)$}}, \text{\ul{$(f_3, l_2)$}}\right\}$\\
\textbf{Instabilitás:} $\left[f_1-l_1\right]$ és \ul{$\left[f_3-l_1\right]$}

\setulcolor{orange}
$M_3=\left\{\text{\ul{$(f_1, l_3)$}}, \text{\ul{$(f_2, l_2)$}}, (f_3, l_1)\right\}$\\
\textbf{Instabilitás:} $\left[f_1-l_2\right]$ és \ul{$\left[f_1-l_1\right]$}

$M_4=\left\{(f_1, l_1), (f_2, l_2), (f_3, l_3)\right\} = M_0$\\
Végtelen ciklus: $M_0, M_1, M_2, M_3, M_4=M_0, M_1, M_2, \cdots$


\subsubsection*{Gale-Shapley algoritmus}
\sethlcolor{orange}
\hl{(zh. feladat)}

\setstcolor{red}
\begin{tabular}{l l}
    $f_1 \rightarrow (\text{\st{$l_3$}}, \text{\st{$l_2$}}, l_5, l_1, l_4)$ & $l_1 \rightarrow (f_3, f_5, f_2, f_1, f_4)$ \\
    $f_2 \rightarrow (\text{\st{$l_1$}}, l_2, l_5, l_3, l_4)$ & $l_2 \rightarrow (f_5, f_2, f_1, f_4, f_3)$ \\
    $f_3 \rightarrow (l_4, l_3, l_2, l_1, l_5)$ & $l_3 \rightarrow (f_4, f_3, f_5, f_1, f_2)$ \\
    $f_4 \rightarrow (\text{\st{$l_1$}}, l_3, l_4, l_2, l_5)$ & $l_4 \rightarrow (f_1, f_2, f_3, f_4, f_5)$ \\
    $f_5 \rightarrow (l_1, l_2, l_4, l_5, l_3)$ & $l_5 \rightarrow (f_2, f_3, f_4, f_1, f_5)$ \\
\end{tabular}

\begin{tabular}{c|c|c|c|c|c|}
    & $l_1$ & $l_2$ & $l_3$ & $l_4$ & $l_5$ \\
    \hline
    1. & $f_2$, $f_4$, \fbox{$f_5$} &  & $f_1$ & $f_3$ &  \\
    2. & $f_5$ & $f_2$ & $f_1$, \fbox{$f_4$} & $f_3$ &  \\
    3. & $f_5$ & \fbox{$f_2$}, $f_1$ & $f_4$ & $f_3$ &  \\
    4. & $f_5$ & $f_2$ & $f_4$ & $f_3$ & $f_1$ \\
\end{tabular}

\textbf{Észrevétel 1:} mindig van egy lány, akinél csak az utolsó napon jelenik meg egy szerenádozó.
Ezt a lányt senki sem húzza ki a preferencialistájáról.\\
\textbf{Észrevétel 2:} a többi lányt legalább 1 fiú nem húzza ki.\\
Kihúzások száma: $n^2-n-(n-1)=(n-1)^2+1$

\textbf{Hány nap alatt fejeződik be biztosan?} Előadás: $n^2+1$ korlát (minden nap legalább 1 lányt kihúznak).\\
\textbf{Észrevétel 1:} $n^2+1-n$\\
\textbf{Észrevétel 2:} $n^2+1-n-(n-1)$ (a végén n-1 mert az első észrevételbeli lányt nem számoljuk még egyszer)\\
\textbf{Összesen:} $n^2+1-n-(n-1)=(n^2-2n+1)+1=(n-1)^2+1$


\textbf{Maximum hány stabil párosítás lehetséges n fiú és n lány között?} Az összes teljes párosítás száma: $n!$.
Gale-Shapley algoritmus: mindig van legalább 1.\\
\textbf{HF.} Ha minden lánynak ugyanaz a preferencialistája, akkor csak 1 létezik.
Észrevétel: néha 1-nél több van.

\begin{tikzpicture}
\node(f1){$f_1 \rightarrow (l_1, l_2)$};
\node(l1)[right = 0.5cm of f1]{$l_1 \rightarrow (f_2, f_1)$};
\node(f2)[below = 0.5cm of f1]{$f_2 \rightarrow (l_2, l_1)$};
\node(l2)[below = 0.5cm of l1]{$l_2 \rightarrow (f_1, f_2)$};
\draw[red] (f1) -- (l1);
\draw[red] (f2) -- (l2);
\draw[green] (f1) -- (l2);
\draw[green] (f2) -- (l1);
\end{tikzpicture}

{\color{red} Stabil párosítás:} minden \textbf{fiú} a preferencialistájának elsőjét kapja.\\
{\color{green} Stabil párosítás:} minden \textbf{lány} a preferencialistájának elsőjét kapja.

\textbf{Meglepetés:}
$2n$ fiú és $2n$ lány van, akár $2^n$ párosítás is lehetséges.\\
\begin{tabular}{l l}
    $f_1 \rightarrow (l_1, l_2, \cdots)$ & $l_1 \rightarrow (\cdots, f_1)$ \\
    $f_2 \rightarrow (l_2, l_1, \cdots)$ & $l_2 \rightarrow (\cdots, f_2)$ \\
    $f_3 \rightarrow (l_3, l_4, \cdots)$ & $l_3 \rightarrow (\cdots, f_3)$ \\
    $f_4 \rightarrow (l_4, l_3, \cdots)$ & $l_4 \rightarrow (\cdots, f_4)$ \\
    $f_{2i-1} \rightarrow (l_{2i-1}, l_{2i}, \cdots)$ & $l_1 \rightarrow (\cdots, f_{2i-1})$ \\
    $f_{2i} \rightarrow (l_{2i}, l_{2i-1}, \cdots)$ & $l_1 \rightarrow (\cdots, f_{2i})$ \\
    \vdots & \vdots \\
    $f_{2n-1} \rightarrow (l_{2n-1}, l_{2n}, \cdots)$ & $l_1 \rightarrow (\cdots, f_{2n-1})$ \\
    $f_{2n} \rightarrow (l_{2n}, l_{2n-1}, \cdots)$ & $l_1 \rightarrow (\cdots, f_{2n})$ \\
\end{tabular}

A sok-sok stabil párosítás:

\raisebox{-.4\height}{
\fbox{
\begin{tikzpicture}
\node(f1){$f_1$};
\node(l1)[right = 0.5cm of f1]{$l_1$};
\node(f2)[below = 0.2cm of f1]{$f_2$};
\node(l2)[below = 0.2cm of l1]{$l_2$};
\draw (f1) -- (l1);
\draw (f2) -- (l2);
\end{tikzpicture}
}
}
vagy
\raisebox{-.4\height}{
\fbox{
\begin{tikzpicture}
\node(f1){$f_1$};
\node(l1)[right = 0.5cm of f1]{$l_1$};
\node(f2)[below = 0.2cm of f1]{$f_2$};
\node(l2)[below = 0.2cm of l1]{$l_2$};
\draw (f1) -- (l2);
\draw (f2) -- (l1);
\end{tikzpicture}
}
}

\raisebox{-.4\height}{
\fbox{
\begin{tikzpicture}
\node(f3){$f_3$};
\node(l3)[right = 0.5cm of f3]{$l_3$};
\node(f4)[below = 0.2cm of f3]{$f_4$};
\node(l4)[below = 0.2cm of l3]{$l_4$};
\draw (f3) -- (l3);
\draw (f4) -- (l4);
\end{tikzpicture}
}
}
vagy
\raisebox{-.4\height}{
\fbox{
\begin{tikzpicture}
\node(f3){$f_3$};
\node(l3)[right = 0.5cm of f3]{$l_3$};
\node(f4)[below = 0.2cm of f3]{$f_4$};
\node(l4)[below = 0.2cm of l3]{$l_4$};
\draw (f3) -- (l4);
\draw (f4) -- (l3);
\end{tikzpicture}
}
}

Minden sorból választunk egyet: $2^n$ lehetőség.\\
\textbf{HF. } az összes így kapott párosítás stabil (nincs instabilitás).

\textbf{Miért lesz ez mind stabil párosítás?}

Ha a felső kockát választottuk, akkor $f_1$ párja a preferencialistáján az első lány, így $f_1$ nem lehet instabilitás fiú tagja.

Ha az alsó kockát választottuk, akkor $f_1$ párja a preferencialistáján a második lány, így $f_1$ csak az $l_1$ lánnyal lehet instabilitásban.
Azonban $f_1$ utolsó $l_1$ preferencialistáján, így akárki is $l_1$ párja, ő jobban tetszik $l_1$-nek, mint $f_1$.
Így $f_1$ most sem lehet instabilitás fiú tagja.

Ez a gondolatmenet minden fiúra alkalmazható, így mind a $2^n$ párosítás stabil.
