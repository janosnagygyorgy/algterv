\section {gyakorlat (2025. szeptember 24.)}
\textbf{Állítás (pareto optimalitás):}
Jelölje $M$ a Gale-Shapley algoritmus által szolgáltatott (fiú-optimális) párosítást.
Ekkor nem létezik olyan $M'$ teljes párosítás (nem stabil sem), ahol minden fiú jobban jár mint $M$-ben.
(Olyat valószínűleg tudunk mutatni, amiben valamelyik fiú jobban jár mint $M$-ben, mivel a Gale-Shapley algoritmus esetén lehet olyan fiú aki nem a listájáról az első lányt kapta.
Ha ez a fiú megkapja a listájának első lányát (és a többi fiú is valamilyen módon kap egy új párt), akkor máris mutattunk egy ilyen párosítást.)

\textbf{Bizonyítás:}
Indirekt tegyük fel, hogy létezik olyan $M'$ teljes párosítás, ahol minden fiú jobban jár, mint $M$-nél.
Nézzük az M-et előállító Gale-Shapley algoritmus lefutását.
Ekkor van olyan $l$ lány, akinek csak az utolsó napon jelenik meg az ablaka alatt szerenádozó, aki végül a párja lesz.
Legyen ez a fiú $f$.
Most $f$ $M'$-beli $l'$ párja jobban tetszik $f$-nek, mint $l$ (spec. $l' \neq l$).
Jelölje $l$ $M$-beli párját $f'$.
Ismét, $f'$ $M'$-beli $l$ párja jobban tetszik $f'$-nek, mint az $M$-beli párja.

Igen ám, de ekkor $f'$ az utolsó nap előtt szerenádozott $l$-nél és kosarat kapott, ellentmondva annak, hogy $l$-nél csak az utolsó nap szerenádozik valaki.

\subsection*{Megoldandó probléma}
$T(n) = a T\left(\frac{n}{b}\right) + f(n)$ alakú rekurziók.

\textbf{Összefésüléses rendezés}\\
$T(n) = 2 T\left(\frac{n}{2}\right) + n \rightarrow T(n) = \Theta(n \log n)$

\textbf{Maximális növekedés}\\
Adott pozitív számoknak egy $A[1:n]$ tömbje.
Keressünk olyan $1 \leq i \leq j \leq n$ indexeket, hogy $A[j] - A[i]$ maximális.

\subsection*{Oszd meg és uralkodj algoritmusok}
Oszd meg és uralkodj algoritmust tervezünk (van más, hatékony módszer is).
\begin{enumerate}
    \item Bontsuk a feladatot két feleakkora méretű feladatra:
    \begin{itemize}
        \item $A[1: \frac{n}{2}]$-ben hol van a maximális növekedés
        \item $A[\frac{n}{2} + 1: n]$-ben hol van a maximális növekedés
    \end{itemize}
    \item Rekurzívan megoldjuk a részfeladatokat:
    \begin{itemize}
        \item $A[1: \frac{n}{2}] \rightarrow 1 \leq i' \leq j' \leq \frac{n}{2}$
        \item $A[\frac{n}{2} + 1: n] \rightarrow \frac{n}{2} + 1 \leq i'' \leq j'' \leq n$
    \end{itemize}
    Egyelemű tömbökre direkt megoldás: a két index megegyezik az elem indexével.
    \item A maximális növekedést adó $i$ és $j$ meghatározása. Három eset van:
    \begin{enumerate}
        \item $A[1: \frac{n}{2}]$-ben (az első felében) van a maximális növekedés $\rightarrow i = i' \text{ és } j = j'$
        \item $A[\frac{n}{2} + 1: n]$-ben (a második felében) van a maximális növekedés $\rightarrow i = i'' \text{ és } j = j''$
        \item $1 \leq i \leq \frac{n}{2} < j \leq n$ (az egyik az első, a másik a második felében). Ekkor $i$-t célszerű úgy választani, hogy $A[i]$ az $A[1: \frac{n}{2}]$ legkisebb értéke, $j$-t pedig úgy, hogy $A[j]$ az $A[\frac{n}{2} + 1: n]$ legnagyobb értéke.
    \end{enumerate}
\end{enumerate}

Nem tudjuk előre, hogy (a), (b) és (c) közül melyik adja az optimumot, ezért mindet megvizsgáljuk és a legkedvezőbbet választjuk.

\subsubsection*{Költség}
$T(n) = \underbrace{2 T (\frac{n}{2})}_{\text{két rekurzív hívás}} + \underbrace{\Theta(n)}_{\text{\parbox{4.1cm}{\centering min $A[1:\frac{n}{2}]$-ben \\ max $A[\frac{n}{2}+1: n]$-ben \\ (a), (b), (c) "közül a legnagyobb"}}}$

Ez egy összefésüléses rendezés rekurzió $\rightarrow T(n) = \Theta(n \log n)$

\clearpage
\textbf{Van hatékonyabb?} Nem meglepő módon igen:

\begin{tikzpicture}[
    squarenodebig/.style={rectangle, draw=black, minimum width=2.4cm, minimum height=0.5cm},
    squarenodesmall/.style={rectangle, draw=black, minimum width=1.2cm, minimum height=0.5cm}
]
\node[squarenodebig](arr_1_1){};
\node[squarenodebig](arr_1_2)[right = 0cm of arr_1_1]{};
\node[squarenodesmall](arr_2_1)[below left = 1.5cm and 0cm of arr_1_1]{};
\node[squarenodesmall](arr_2_2)[right = 0cm of arr_2_1]{};
\node[squarenodesmall](arr_2_4)[below right = 1.5cm and 0cm of arr_1_2]{};
\node[squarenodesmall](arr_2_3)[left = 0cm of arr_2_4]{};

\draw[thick, decoration={brace, raise=0.3cm}, decorate] (arr_1_1.west) -- (arr_1_1.east);
\draw[thick, decoration={brace, raise=0.3cm}, decorate] (arr_1_2.west) -- (arr_1_2.east);
\draw[thick, decoration={brace, raise=0.3cm}, decorate] (arr_2_1.west) -- (arr_2_1.east);
\draw[thick, decoration={brace, raise=0.3cm}, decorate] (arr_2_2.west) -- (arr_2_2.east);
\draw[thick, decoration={brace, raise=0.3cm}, decorate] (arr_2_3.west) -- (arr_2_3.east);
\draw[thick, decoration={brace, raise=0.3cm}, decorate] (arr_2_4.west) -- (arr_2_4.east);

\node[above = 0.2cm of arr_1_1]{$min$};
\node[above = 0.2cm of arr_1_2]{$max$};
\node[above = 0.2cm of arr_2_1]{$min^{I}$};
\node[above = 0.6cm of arr_2_1]{$max^{I}$};
\node(min2)[above = 0.6cm of arr_2_2]{$min^{II}$};
\node[above = 0.2cm of arr_2_2]{$max^{II}$};
\node[above = 0.2cm of arr_2_3]{$min^{III}$};
\node[above = 0.6cm of arr_2_3]{$max^{III}$};
\node(min4)[above = 0.6cm of arr_2_4]{$min^{IV}$};
\node[above = 0.2cm of arr_2_4]{$max^{IV}$};

\draw[->] (arr_1_1.south) -- (min2.north west);
\draw[->] (arr_1_2.south) -- (min4.north west);
\end{tikzpicture}

A rekurzív hívásokba süllyesztve a min és max kiválasztást $\Theta(n)$ $\Theta(1)$-re csökken a rekurzióban.

Zárt formula erre: $T(n) = 2T (\frac{n}{2}) + 1$\\
\textbf{HF. } direkt számolás (önmagába helyettesítés)

Még egy érdekes dolog:\\
$T(n) = T(\frac{n}{2}) + 1 \rightarrow T(n) = \Theta(\log n)$\\
$T(n) = T(\frac{n}{2}) + n \rightarrow T(n) = \Theta(n)$

\subsection{Mester-tétel}
Könyv: \url{../konyv.pdf#page=10}\\
$f(n) \leftrightarrow n^{\log_b a}$

\begin{enumerate}
    \item
    \begin{itemize}
        \item $T(n) = aT\left(\frac{n}{b}\right) + f(n) = 9T\left(\frac{n}{3}\right) + n$
        \item $n^{\log_b a} = n^{\log_3 9} = n^2$ polinomiálisan nagyobb, mint $f(n)$
        \item $[f(n) = n = O(n^{2-\mathcal{E}})$, például $\mathcal{E}=\frac{1}{2}$ esetén$]$
        \item Mester tétel első eset $\rightarrow T(n) = \Theta(n^{log_b a}) = \Theta(n^2)$
    \end{itemize}
    \item
    \begin{itemize}
        \item $T(n) = T\left(\frac{2n}{3}\right) + 1$ ($a=1$, $b=\frac{3}{2}$, $f(n) = 1$)
        \item $n^{log_b a} = n^{log_{\frac{3}{2}} 1} = n^0 = 1$ aszimptotikusan megegyezik $f(n)$-nel.
        \item $[f(n) = 1 = \Theta(n^0) = \Theta(1)]$
        \item Mester tétel második eset $\rightarrow T(n) = \Theta(n^{\log_b a} \log n) = \Theta(\log n)$ 
    \end{itemize}
        \item
    \begin{itemize}
        \item $T(n) = 3T\left(\frac{n}{4}\right) + n \log n$ ($a=3$, $b=4$, $f(n) = n \log n$)
        \item $n^{log_b a} = n^{log_4 3} \approx n^{0,793}$ ($\log_4 3 < log_4 4 = 1$)
        \item $f(n)$ polinomiálisan nagyobb, mint $n^{0,793}$
        \item $f(n) = n \log n = \Omega(n^{0,793 + \mathcal{E}})$ pl. $\mathcal{E}=0,1$ esetén
        \item A Mester tétel harmadik esetének néz ki, de ahhoz hogy tényleg az legyen, még egy dolgot ellenőrizni kell:
        \begin{itemize}[label=$\diamond$]
            \item a $f\left(\frac{n}{b}\right) \leq c f(n)$ alkalmas $c < 1$ konstanssal
            \item $3 f\left(\frac{n}{4}\right) = 3 \left(\frac{n}{4} \log \frac{n}{4}\right) = \frac{3}{4} n \log \frac{n}{4} \leq \frac{3}{4} n \log n$
            \item $c = \frac{3}{4}$ megfelelő
            \item Így tényleg mester tétel harmadik eset $\rightarrow T(n) = \Theta(f(n)) = \Theta(n \log n)$
        \end{itemize}
    \end{itemize}
\end{enumerate}

\textbf{HF. } mindenféle ilyen rekurziók:
\begin{itemize}
    \item $T(n) = 2T\left(\frac{n}{2}\right) + n^3$
    \item $T(n) = 2T\left(\frac{n}{2}\right) + n^2$
    \item $T(n) = 2T\left(\frac{n}{2}\right) + n \log n$
\end{itemize}
