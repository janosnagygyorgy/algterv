\section{gyakorlat (2025. október 1.)}
\textbf{"Oszd meg és uralkodj"}
\begin{enumerate}
    \item részproblémákra bontjuk
    \item a részproblémákat megoldjuk
    \item ezek eredményéből kiszámoljuk a végeredményt
\end{enumerate}

\subsubsection*{Inverziószámok keresése/számlálása}
Legyen $A[1..n]$ egy tömb egyedi számokkal. Az $(i, j)$ indexpár inverzió, ha $1 \leq i < j \leq n$ és $A[i] > A[j]$.
Számoljuk meg, hány inverzió van a tömbben.

\textbf{Naiv módszer:}\\
Minden indexpárt ellenőrzünk: $O(n^2)$.

\textbf{Oszd meg és uralkodj:}\\
(könyv: \url{../konyv.pdf#page=8})
\begin{enumerate}
    \item Részproblémákra bontjuk: külön keressük az inverziókat $A[1:\frac{n}{2}]$ és $A[\frac{n}{2}+1:n]$-ben
    \item A részproblémák megoldása (rekurzívan): ha a részprobléma 1 elemű, akkor nincs inverzió, adjunk vissza 0-t
    \item Válaszok egyesítése: keressük az összes inverziót.
    \begin{enumerate}
        \item $1 \leq i < j \leq \frac{n}{2} \rightarrow A[1: \frac{n}{2}]$ részprobléma
        \item $\frac{n}{2} + 1 \leq i < j \leq n \rightarrow A[\frac{n}{2} + 1: n]$ részprobléma
        \item $1 \leq i \leq \frac{n}{2} < j \leq n \rightarrow \; ?$
    \end{enumerate}
\end{enumerate}

Ötlet: rendezzünk menet közben (merge sort).
Ha rendezett a két résztömb, könnyebb a (c) esetet ellenőrizni.
Két eset van:
\begin{itemize}
    \item $A[i] < A [j] \rightarrow$ nincs inverzió. Mivel rendezett a tömb, ezért minden ($j \leq k \leq n$) $k$ index se lesz inverzió.
    \item $A[i] > A[j] \rightarrow$ inverzió. Mivel rendezettek a tömbök, ezért minden ($i \leq l \leq \frac{n}{2}$) $l$ index is inverzió lesz j-vel.
\end{itemize}

\textbf{Végeredmény:} a két részprobléma megoldásai + a köztük végzett ellenőrzés eredménye.
Futási idő: $\mathcal{O}(n \cdot \log n)$ mint a merge sort.

\subsubsection*{Többségi elem keresése}
Egy elem többségi elem egy $A[1:n]$ tömbben, ha $\frac{n}{2}$-nél többször fordul elő benne (pl. minimum 6/10, minimum 4/7, stb.).
Ha van többségi elem egy tömbben, akkor biztosan egy van.
Keressük meg a többségi elemet, amennyiben van ilyen.

\textbf{Naiv módszer:}\\
Minden elemet megszámolunk: $O(n^2)$.

\textbf{Oszd meg és uralkodj:}
\begin{enumerate}
    \item Részproblémák:
    \begin{itemize}
        \item Ötlet: ha $x$ többségi elem $A[1:n]$ tömbben, akkor (mivel $\frac{n}{2}$-nél többször fordul elő) többségi elem lesz $A[1:\frac{n}{2}]$-ben vagy $A[\frac{n}{2}+1:n]$-ben.
        \item Szóval a részproblémák: vegyük a tömb egyik ($A[1:\frac{n}{2}]$) és másik ($A[\frac{n}{2}+1:n]$) felét. 
    \end{itemize}
    \item Részproblémák megoldása rekurzívan:
    \begin{itemize}
        \item Ha 1 elemű a tömb, akkor az az elem a többségi elem.
        \item $A[1:\frac{n}{2}] \rightarrow m'$, $A[\frac{n}{2}+1:n] \rightarrow m''$ ($m'$ és $m''$ lehet üres is (ha nincs többségi elem az adott résztömbben))
        \item $m'$ és $m''$ közül valamelyik az egész tömb többségi eleme is lesz.
    \end{itemize}
    \item Számoljuk meg az előfordulásokat, hogy $m'$ és $m''$ hányszor szerepel $A[1:n]$-ben.
    \begin{itemize}
        \item Ha $m'$ többször fordul elő, mint $m''$, akkor $m'$ lesz a többségi elem.
        \item Ha $m''$ többször fordul elő, mint $m'$, akkor $m''$ lesz a többségi elem.
        \item Ha $m'$ és $m''$ ugyanannyiszor fordul elő, akkor nincs többségi elem.
    \end{itemize}
\end{enumerate}

\textbf{Futási idő:} $\mathcal{O}(n \log n)$

\textbf{Tudunk-e erre a problémára hatékonyabb (lineáris) algoritmust?}
Igen, de nem oszd meg és uralkodj.

\begin{tikzpicture}[
    squarenode/.style={rectangle, draw=black, minimum width=0.3cm, minimum height=0.3cm},
]
\node[squarenode](a1){};
\node[squarenode](a2)[right = 0cm of a1]{};
\node[squarenode](a3)[right = 0cm of a2]{};
\node[squarenode](a4)[right = 0cm of a3]{};
\node[squarenode](a5)[right = 0cm of a4]{};
\node[squarenode](a6)[right = 0cm of a5]{};
\node[squarenode](a7)[right = 0cm of a6]{};
\node[squarenode](a8)[right = 0cm of a7]{};
\node[squarenode](a9)[right = 0cm of a8]{};
\node[squarenode](a10)[right = 0cm of a9]{};
\node[right = 0.2cm of a10]{$A[1:n]$};
\end{tikzpicture}

Ha tudjuk, hogy $x$ többségi elem $A[1:n]$ és $A[1] \neq A[2]$, akkor többségi elem $A[3:n]$-ben.
$A[1:10]$, ha $x$ többségi elem, akkor legalább 6-szor van jelen $A[3:10]$-ben.

\textbf{Ötlet:} rendezzük át az $A[1:n]$ tömböt úgy, hogy párokat kapjunk a tömb elején, amik nem azonos értékek.

\begin{tikzpicture}[
    squarenode/.style={rectangle, draw=black, minimum width=0.3cm, minimum height=0.3cm},
]
\node[squarenode](a1){};
\node[squarenode](a2)[right = 0cm of a1]{};
\node[squarenode](a3)[right = 0cm of a2]{};
\node[squarenode](a4)[right = 0cm of a3]{};
\node[squarenode](a5)[right = 0cm of a4]{};
\node[squarenode](a6)[right = 0cm of a5]{};
\node[squarenode](a7)[right = 0cm of a6]{};
\node[squarenode](a8)[right = 0cm of a7]{};
\node[squarenode](a9)[right = 0cm of a8]{};
\node[squarenode](a10)[right = 0cm of a9]{};
\draw[thick, decoration={brace, raise=0.05cm, mirror}, decorate] (a1.south west) -- (a4.south east);
\draw[thick, decoration={brace, raise=0.05cm}, decorate] (a5.north west) -- (a10.north east);
\node[below right = 0.2cm and -0.35cm of a1]{különböző elemek};
\node[below right = 0.6cm and 0.5cm of a1]{párjai};
\node[above = 0.2cm of a8]{minden elem ugyanaz};
\node[below right = -0.1cm and 0cm of a10](majority_text){ez csak többségi elem lehet};
\draw[->] (majority_text.west) -- (a8.south);
\end{tikzpicture}

\textbf{Hogyan?}

\begin{tikzpicture}[
    squarenode/.style={rectangle, draw=black, minimum width=0.3cm, minimum height=0.3cm},
]
\node[squarenode](a1){};
\node[squarenode](a2)[right = 0cm of a1]{};
\node[squarenode](a3)[right = 0cm of a2]{};
\node[squarenode](a4)[right = 0cm of a3]{};
\node[squarenode](a5)[right = 0cm of a4]{};
\node[squarenode](a6)[right = 0cm of a5]{};
\node[squarenode](a7)[right = 0cm of a6]{};
\node[squarenode](a8)[right = 0cm of a7]{};
\node[squarenode](a9)[right = 0cm of a8]{};
\node[squarenode](a10)[right = 0cm of a9]{};
\node(i)[below = 0.3cm of a3]{i};
\node(j)[below = 0.3cm of a7]{j};
\draw[->] (i) -- (a3);
\draw[->] (j) -- (a7);
\end{tikzpicture}

Tegyük fel, hogy $A[j]$-nél vagyunk, és $A[1:i]$ a párok sora, míg $A[i+1:j]$ homogén.
Ekkor ha $A[j] = A[i+1]$, akkor a homogén részhez hozzávesszük $A[j]$-t.
Ha viszont $A[j] \neq A[i+1]$, akkor megcseréljük $A[j]$-t $A[i+2]$-vel, és ezzel növeljük a párok számát (az első rész 2-vel hosszabb lesz).
Innentől a homogén rész az $A[i+3:j]$.
A rendezés végén ha marad homogén rész, akkor az a többségi elem.

\textbf{Futási idő:} mivel egyszer megyünk végig, ez lineáris, $O(n)$.

\subsubsection*{Mester tétel gyakorlás}

\begin{itemize}
    \item $T(n) = 2T(\frac{n}{2}) + n^3$
    \item $a = 2$, $b = 2$, $f(n) = n^3$
    \item $n^{\log_b a} = n^{\log_2 2} = n^1 = n$
    \item $f(n) = n^3 > n$
\end{itemize}

3. eset:
$f(n) = \Omega(n^{\log_b a + \mathcal{E}}) \rightarrow \mathcal{E}=2$ \checkmark

Regularitás:
\begin{flalign*}
    a f(\frac{n}{b}) &\leq c f(n)&&\\
    2 f(\frac{n}{2}) &\leq c f(n)&&\\
    2 \frac{n^3}{8} &\leq c n^3&&\\
    f(\frac{n^3}{8}) &\leq c n^3&&\\
    \frac{n^3}{4} &\leq c n^3
\end{flalign*}

$c < 1$, $c \geq \frac{1}{4}$ \checkmark

$T(n) = \Theta(f(n)) = \Theta(n^3)$
