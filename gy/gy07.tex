\section{gyakorlat (2025. október 22.)}
\subsection{Dinamikus programozás}
\subsubsection*{Mátrixok optimális zárójelezése}
$A_1 A_2 \cdots A_n$\\
A szorzat nem függ a zárójelezéstől, de a kiszámításhoz szükséges elemi szorzások száma általában nagyon is.

\sethlcolor{orange}
\textbf{Példa} \hl{(zh. feladat)}:

Mátrixok:
\begin{flalign*}
    &A_1: 25 \times 30 &&\\
    &A_2: 30 \times 10 &&\\
    &A_3: 10 \times 5 &&\\
    &A_4: 5 \times 10 &&\\
    &A_5: 10 \times 15 &&\\
    &A_6: 15 \times 20 &&
\end{flalign*}
Dimenziók ($A_1 A_2 A_3 A_4 A_5 A_6$ sorrend alapján sorban):
\begin{flalign*}
    &q_0 = 25 &&\\
    &q_1 = 30 &&\\
    &q_2 = 10 &&\\
    &q_3 = 5  &&\\
    &q_4 = 10 &&\\
    &q_5 = 15 &&\\
    &q_6 = 20 &&
\end{flalign*}
Feladat: $A_1 A_2 A_3 A_4 A_5 A_6$ optimális zárójelezése.

Megoldás: két $6 \times 6$-os táblázat

\begin{tabular}{c c}
    \includegraphics[width=6cm]{gy/img/gy07_table_l.pdf} & \includegraphics[width=6cm]{gy/img/gy07_table_k.pdf} \\
    $l[i,j]$ & $k[i,j]$ ("az a bizonyos $k$") \\
\end{tabular}

\begin{tabular}{l l}
    $l[1,1] = 0$ & $l[1,2] = min\left\{l[1,1] + l[2,2] + q_0 + q_1 + q_2 \right\} = 0 + 0 + 25 \times 30 \times 10 = 7500$ \\
    $l[2,2] = 0$ & $l[2,3] = 30 \times 10 \times 5 = 1500$  \\
    $l[3,3] = 0$ & $l[3,4] = 10 \times 5 \times 10 = 500$   \\
    $l[4,4] = 0$ & $l[4,5] = 5 \times 10 \times 15 = 750$   \\
    $l[5,5] = 0$ & $l[5,6] = 10 \times 15 \times 20 = 3000$ \\
    $l[6,6] = 0$ &                                          \\
\end{tabular}

\textbf{Általános formula:}\\
$l[i,j] = \underset{i \leq k \leq j-1}{min} \{ l[i,k] + l[k+1, j] + \underbrace{q_{i-1} q_k q_j}_{\text{dimenziók}} \}$\\
$k[i,j] = $ az a k, ahol a fenti zárójeles összeg a legkisebb.

\clearpage
\textbf{1. lépés: "átlós kitöltés":} amikor egy adott $l[i,j]$-t számítjuk, már minden $l[i,k]$ (a sorban előtte lévők) és $l[k+1, j]$ (az oszlopban alatta lévők) ismert.

\begin{flalign*}
&l[1,3] = min 
\begin{Bmatrix}
    l[1,\text{\fbox{$1$}}] + l[2,3] + 25 \cdot 30 \cdot 5 = \text{\fbox{$5250$}} \\
    l[1,2] + l[3,3] + 25 \cdot 10 \cdot 5 = 8750
\end{Bmatrix}&&
\end{flalign*}

\begin{flalign*}
&l[2,4] = min 
\begin{Bmatrix}
    l[2,2] + l[3,4] + 30 \cdot 10 \cdot 10 = 3500 \\
    l[2,\text{\fbox{$3$}}] + l[4,4] + 30 \cdot 5 \cdot 10 = \text{\fbox{$3000$}}
\end{Bmatrix}&&
\end{flalign*}

\begin{flalign*}
&l[3,5] = min 
\begin{Bmatrix}
    l[3,\text{\fbox{$3$}}] + l[4,5] + 10 \cdot 5 \cdot 15 = \text{\fbox{$1500$}} \\
    l[3,4] + l[5,5] + 10 \cdot 10\cdot 15 = 2000
\end{Bmatrix}&&
\end{flalign*}

\begin{flalign*}
&l[4,6] = min 
\begin{Bmatrix}
    l[4,4] + l[5,6] + 5 \cdot 10 \cdot 20 = 4000 \\
    l[4,\text{\fbox{$5$}}] + l[6,6] + 5 \cdot 15 \cdot 20 = \text{\fbox{$2250$}}
\end{Bmatrix}&&
\end{flalign*}

\begin{flalign*}
&l[1,4] = min 
\begin{Bmatrix}
    l[1,1] + l[2,4] + 25 \cdot 30 \cdot 10 = 10500 \\
    l[1,2] + l[3,4] + 25 \cdot 10 \cdot 10 = 10500 \\
    l[1,\text{\fbox{$3$}}] + l[4,4] + 25 \cdot 5 \cdot 10 = \text{\fbox{$6500$}}
\end{Bmatrix}&&
\end{flalign*}
Ha minden sor egyenlő, akkor bármelyiket választhatjuk minimumnak, majd a backtracking során látjuk, hogy ekkor több optimális zárójelezés is létezik.

\begin{flalign*}
&l[2,5] = min 
\begin{Bmatrix}
    l[2,2] + l[3,5] + 30 \cdot 10 \cdot 15 = 6000 \\
    l[2,\text{\fbox{$3$}}] + l[4,5] + 30 \cdot 5 \cdot 15 = \text{\fbox{$4500$}} \\
    l[2,4] + l[5,5] + 30 \cdot 10 \cdot 15 = 7500
\end{Bmatrix}&&
\end{flalign*}

\begin{flalign*}
&l[3,6] = min 
\begin{Bmatrix}
    l[3,\text{\fbox{$3$}}] + l[4,6] + 10 \cdot 5 \cdot 20 = \text{\fbox{$3250$}} \\
    l[3,4] + l[5,6] + 10 \cdot 10 \cdot 20 = 5500 \\
    l[3,5] + l[6,6] + 10 \cdot 15 \cdot 20 = 4500
\end{Bmatrix}&&
\end{flalign*}

\begin{flalign*}
&l[1,5] = min 
\begin{Bmatrix}
    l[1,1] + l[2,5] + 25 \cdot 30 \cdot 15 = 15750 \\
    l[1,2] + l[3,5] + 25 \cdot 10 \cdot 15 = 12750 \\
    l[1,\text{\fbox{$3$}}] + l[4,5] + 25 \cdot 5 \cdot 15 = \text{\fbox{$7875$}} \\
    l[1,4] + l[5,5] + 25 \cdot 10 \cdot 15 = 10250
\end{Bmatrix}&&
\end{flalign*}

\begin{flalign*}
&l[2,6] = min 
\begin{Bmatrix}
    l[2,2] + l[3,6] + 30 \cdot 10 \cdot 20 =  9250 \\
    l[2,\text{\fbox{$3$}}] + l[4,6] + 30 \cdot 5 \cdot 20 = \text{\fbox{$6750$}} \\
    l[2,4] + l[5,6] + 30 \cdot 10 \cdot 20 = 12000 \\
    l[2,5] + l[6,6] + 30 \cdot 15 \cdot 20 = 13500
\end{Bmatrix}&&
\end{flalign*}

\begin{flalign*}
&l[1,6] = min 
\begin{Bmatrix}
    l[1,1] + l[2,6] + 25 \cdot 30 \cdot 20 =  21750 \\
    l[1,2] + l[3,6] + 25 \cdot 10 \cdot 20 = 15750 \\
    l[1,\text{\fbox{$3$}}] + l[4,6] + 25 \cdot 5 \cdot 20 = \text{\fbox{$10000$}} \\
    l[1,4] + l[5,6] + 25 \cdot 10 \cdot 20 = 14500\\
    l[1,5] + l[6,6] + 25 \cdot 15 \cdot 20 = 15375
\end{Bmatrix}&&
\end{flalign*}

\clearpage
\textbf{2. lépés: "korábbi szorzások":}\\
Optimális zárójelezése $A_1 A_2 A_3 A_4 A_5 A_6$-nak

$A_1 A_2 A_3 A_4 A_5 A_6 \rightarrow k[1,6] = 3 \rightarrow {\color{orange}(}A_1 A_2 A_3{\color{orange})}{\color{orange}(}A_4 A_5 A_6{\color{orange})}$

Optimális zárójelezése $A_1 A_2 A_3$-nak és $A_4 A_5 A_6$-nak
\begin{flalign*}
    & A_1 A_2 A_3 \rightarrow k[1,3] = 1 \rightarrow A_1 {\color{blue}(}A_2 A_3{\color{blue})} &&\\
    & A_4 A_5 A_6 \rightarrow k[4,6] = 5 \rightarrow {\color{green}(}A_4 A_5{\color{green})} A_6 &&
\end{flalign*}
Szumma szummárum: ${\color{orange}(}A_1 {\color{blue}(}A_2 A_3{\color{blue})}{\color{orange})}{\color{orange}(}{\color{green}(}A_4 A_5{\color{green})} A_6{\color{orange})}$
