\section{gyakorlat (2025. november 19.)}
\subsection{Előadás végéhez némi háttér}

$6^{1589}$ mennyi maradékot ad 31-gyel osztva?
Naiv módszer: $\underbrace{6 \cdot 6 \cdot 6 \cdot 6 \cdot \cdots \cdot 6}_{\text{1589-szer}}$ \rightarrow 1588 szorzás

31-gyel oszthatóság

\begin{tikzpicture}
\node(s1){$6$};
\node(m1)[right=0cm of s1]{$\cdot$};
\node(s2)[right=0cm of m1]{$6$};
\node(m2)[right=0cm of s2]{$\cdot$};
\node(s3)[right=0cm of m2]{$6$};
\node(m3)[right=0cm of s3]{$\cdot$};
\node(s4)[right=0cm of m3]{$6$};
\node(m4)[right=0cm of s4]{$\cdot$};
\node(s5)[right=0cm of m4]{$6$};
\node(m5)[right=0cm of s5]{$\cdot$};
\node(c)[right=0cm of m5]{$\cdots$};
\node(m6)[right=0cm of c]{$\cdot$};
\node(s6)[right=0cm of m6]{$6$};

\draw[decorate,decoration={brace, mirror}] (s1.south west) -- (s2.south east);
\draw[decorate,decoration={brace, mirror, raise=1cm}] (s1.south west) -- (s3.south east);
\draw[decorate,decoration={brace, mirror, raise=1.6cm}] (s1.south west) -- (s4.south east);
\draw[decorate,decoration={brace, mirror, raise=2.2cm}] (s1.south west) -- (s5.south east);

\node[below right =0.1cm and -0.4cm of s1](r1_1){$36 \mod 31 = 5$};
\node[below right =0.6cm and -0.4cm of m1]{$5$};
\node[below right =1.1cm and -0.4cm of s1]{$30 \mod 31 = 30$};
\node[below right =1.7cm and -0.4cm of s1]{$30 \cdot 6 = 180 \mod 31 = 25$};
\node[below right =2.3cm and -0.4cm of s1]{$25 \cdot 6 = 150 \mod 31 = 26$};
\end{tikzpicture}

Hogyan lehet ezt felgyorsítani?

$6^{1589} \rightarrow$ kitevő kettes számrendszerben

\begin{tabular}{lll}
  \begin{tabular}{c@{\,}c@{\,}c@{\,}c@{\,}c@{\,}c@{\,}c@{\,}c@{\,}c@{\,}c}
  1&5&8&9&:&2&=&7&9&4\\
  &1&8& & & & & & & \\
  & &0&9& & & & & & \\
  & & &\fbox{1}& & & & & & \\
  \end{tabular}
  &
  \begin{tabular}{c@{\,}c@{\,}c@{\,}c@{\,}c@{\,}c@{\,}c@{\,}c@{\,}c}
  7&9&4&:&2&=&3&9&7\\
  1&9& & & & & & & \\
  &1&4& & & & & & \\
  & &\fbox{0}& & & & & & \\
  \end{tabular}
  &
  \begin{tabular}{c@{\,}c@{\,}c@{\,}c@{\,}c@{\,}c@{\,}c@{\,}c@{\,}c}
  3&9&7&:&2&=&1&9&8\\
  &1&9& & & & & & \\
  & &1&7& & & & & \\
  & & &\fbox{1}& & & & & \\
  \end{tabular}
  \\\\

  \begin{tabular}{c@{\,}c@{\,}c@{\,}c@{\,}c@{\,}c@{\,}c@{\,}c}
  1&9&8&:&2&=&9&9\\
  &1&8& & & & & \\    
  & &\fbox{0}& & & & & \\    
  \end{tabular}
  &
  \begin{tabular}{c@{\,}c@{\,}c@{\,}c@{\,}c@{\,}c@{\,}c}
  9&9&:&2&=&4&9\\
  1&9& & & & & \\    
  &\fbox{1}& & & & & \\    
  \end{tabular}
  &
  \begin{tabular}{c@{\,}c@{\,}c@{\,}c@{\,}c@{\,}c@{\,}c}
  4&9&:&2&=&2&4\\
  0&9& & & & & \\     
  &\fbox{1}& & & & & \\     
  \end{tabular}
  \\\\

  \begin{tabular}{c@{\,}c@{\,}c@{\,}c@{\,}c@{\,}c@{\,}c}
  2&4&:&2&=&1&2\\
  0&4& & & & & \\    
  &\fbox{0}& & & & & \\    
  \end{tabular}
  &
  \begin{tabular}{c@{\,}c@{\,}c@{\,}c@{\,}c@{\,}c}
  1&2&:&2&=&6\\
  &\fbox{0}& & & & \\   
  \end{tabular}
  &
  \begin{tabular}{c@{\,}c@{\,}c@{\,}c@{\,}c}
  6&:&2&=&3\\
  \fbox{0}& & & & \\    
  \end{tabular}
  \\\\

  \begin{tabular}{c@{\,}c@{\,}c@{\,}c@{\,}c}
  3&:&2&=&1\\
  \fbox{1}& & & & \\    
  \end{tabular}
  &
  \begin{tabular}{c@{\,}c@{\,}c@{\,}c@{\,}c}
  1&:&2&=&0\\
  \fbox{1}& & & & \\
  \end{tabular}
  & \\
\end{tabular}

\begin{tabular}{ccccccccccc}
  1 & 1 & 0 & 0 & 0 & 1 & 1 & 0 & 1 & 0 & 1 \\
  \ul{1024} & \ul{512} & \ul{256} & \ul{128} & \ul{64} & \ul{32} & \ul{16} & \ul{8} & \ul{4} & \ul{2} & \ul{1} \\
\end{tabular}

1024 + 512 + 32 + 16 + 4 + 1 = 1589 (ahol 1-esek vannak)

Észrevétel: $\mathcal{O}(\log \text{kitevő})$ "elemi" művelet a kette számrendszerre átírás

Ezek után jön a piszkos trükk:
$6^{1589} = 6^{1024 + 512 + 32 + 16 + 4 + 1} = 6^{1024} \cdot 6^{512} \cdot 6^{32} \cdot 6^{16} \cdot 6^{4} \cdot 6^{1}$
\begin{flalign*}
  &6 \mod 31 = 6 &&\\
  &6^2 \mod 31 = 5 &&\\
  &6^4 = (6^2)^2 \mod 31 = 5^2 \mod 31 = 25 \mod 31 = 25 &&\\
  &6^8 = (6^4)^2 \mod 31 = 25^2 \mod 31 = 225 \mod 31 = 8 &&\\
  &6^16 = (6^8)^2 \mod 31 = 8^2 \mod 31 = 64 \mod 31 = 2 &&\\
  &6^32 = (6^16)^2 \mod 31 = 2^2 \mod 31 = 4 &&\\
  &6^64 = (6^32)^2 \mod 31 = 4^2 \mod 31 = 16 &&\\
  &6^128 = (6^64)^2 \mod 31 = 16^2 \mod 31 = 256 \mod 31 = 8 &&\\
  &6^256 = (6^128)^2 \mod 31 = 8^2 \mod 31 = 64 \mod 31 = 2 &&\\
  &6^512 = (6^256)^2 \mod 31 = 2^2 \mod 31 = 4 &&\\
  &6^1024 = (6^512)^2 \mod 31 = 4^2 \mod 31 = 16 &&
\end{flalign*}

Így $6^{1024} \cdot 6^{512} \cdot 6^{32} \cdot 6^{16} \cdot 6^{4} \cdot 6^{1} \mod 31 =$\\
$16 \cdot 4 \cdot 4 \cdot 2 \cdot 25 \cdot 6 \mod 31$

\begin{tikzpicture}
\node(s1){$16$};
\node(m1)[right=0cm of s1]{$\cdot$};
\node(s2)[right=0cm of m1]{$4$};
\node(m2)[right=0cm of s2]{$\cdot$};
\node(s3)[right=0cm of m2]{$4$};
\node(m3)[right=0cm of s3]{$\cdot$};
\node(s4)[right=0cm of m3]{$2$};
\node(m4)[right=0cm of s4]{$\cdot$};
\node(s5)[right=0cm of m4]{$25$};
\node(m5)[right=0cm of s5]{$\cdot$};
\node(s6)[right=0cm of m5]{$6$};

\draw[decorate,decoration={brace, mirror}] (s1.south west) -- (s2.south east);
\draw[decorate,decoration={brace, mirror, raise=0.5cm}] (s1.south west) -- (s3.south east);
\draw[decorate,decoration={brace, mirror, raise=1cm}] (s1.south west) -- (s4.south east);
\draw[decorate,decoration={brace, mirror, raise=1.6cm}] (s1.south west) -- (s5.south east);
\draw[decorate,decoration={brace, mirror, raise=2.2cm}] (s1.south west) -- (s6.south east);

\node[below=0.1cm of m1]{$2$};
\node[below=0.6cm of s2]{$8$};
\node[below=1.1cm of m2]{$16$};
\node[below=1.7cm of s3]{$16 \cdot 25 = 400 \rightarrow 28$};
\node[below=2.3cm of m3]{$28 \cdot 6 = 168 \rightarrow 13$};
\end{tikzpicture}

$\Rightarrow 6^{1589} \equiv \underbrace{13}_{\text{maradék}} \mod 31$

Ez megint $\mathcal{O}(\log \text{kitevő})$ elemi művelet, azaz $\Theta(\text{kitevő})$ elemi művelet $\mathcal{O}(\log \text{kitevő})$ elemi műveletre csökkent.

Mi ennek a jelentősége?\\
Kitevő, mint input mérete $\mathcal{O}(\log \text{kitevő})$.
Ehhez képest $\Theta(\text{kitevő})$ exponenciálisan nagy.
Így a naiv algoritmus exponenciális a bemenet méretében.
A "piszkos trükk" ezt viszi le lineárisra (a bemenet méretében).

Előadásra visszatérve: $a_1, a_2, \cdots, a_n \longrightarrow B$

Részhalmaz-összeg: mindenkit egyszer használhatunk\\
Pénzváltás: többször is.
Nyilván
$a_1$-et maximum $\left\lfloor \frac{B}{a_1} \right\rfloor$,
$a_2$-t maximum $\left\lfloor \frac{B}{a_2} \right\rfloor$, stb. alkalommal.


Pénzváltás naiv visszavezetése a részhalmaz-összegre:\\
$\underbrace{a_1, \cdots, a_1}_{\left\lfloor \frac{B}{a_1} \right\rfloor}, \underbrace{a_2, \cdots, a_2}_{\left\lfloor \frac{B}{a_2} \right\rfloor}, \cdots, \underbrace{a_n, \cdots, a_n}_{\left\lfloor \frac{B}{a_n} \right\rfloor} \rightarrow B$

Probléma: az input mérete a bemenet méretében, különös tekintettel B-re exponenciálisan nagy lett.
Nyilván azok az $a_i$-k, amelyek nagyobban $B$-nél nem sok vizet zavarnak, így a bemenet mérete $\mathcal{O}(n \log B)$ (nagyon precízen $\mathcal{O}((n + 1) \log B)$) a részhalmaz-összegnél.

Ha most $n$ lineáris $B$-ben, akkor ez $B \cdot \log B$ nagyságrendű, ami $2^{\log_2 B} \cdot B$, és ez exponenciálisan nagy $B$ méretében.

Ez azt jelenti, hogy hiába is lenne mondjuk egy kvadratikus algoritmusunk a részhalmaz-összegre (senki nem ismer ilyet), az exponenciális ideig futna a pénzváltásból előállított inputon.

A piszkos trükk itt:
$\underbrace{a_1, \cdots, a_1}_{\left\lfloor \frac{B}{a_1} \right\rfloor} \rightarrow$ amit ezekből elő tudunk állítani összegként, azt mind elő tudjuk állítani sokkal kevesebb számból is: $a_1, 2 a_1, 4 a_1, 8 a_1, \cdots, 2^{\left\lfloor \log_2 B \right\rfloor} a_1$
Ezzel a $\left\lfloor \frac{B}{a_1} \right\rfloor$ db szám lecsökken $\left\lfloor \log_2 B \right\rfloor$ darabra.

Így a transzformált bemenet mérete már nem exponenciális $\log B$-hez képest (igazából lineáris)!

\clearpage
\subsection{"Sztringológia" (folytatás)}
\subsubsection*{Optimális szekvenciaillesztés}
\textbf{Példa} \hl{(zh. feladat)}:\\
Kitöltés menete:
\begin{itemize}
    \item Ha megegyeznek: min((felette lévő + 3), (átlós elem + 0), (balra lévő + 3))
    \item Ha különbözőek: min((felette lévő + 3), (átlós elem + 5), (balra lévő + 3))
\end{itemize}

\begin{tabular}{cccccccccccccc}
  &                         &    & B & A & A                     & C                       & B                       & B                     & A                       & A                       & C  & B  & C                       \\
  &                         & 0  & 1 & 2 & 3                     & 4                       & 5                       & 6                     & 7                       & 8                       & 9  & 10 & 11                      \\ \cline{3-14}
  & \multicolumn{1}{c|}{0}  & 0  & 3 & 6 & 9                     & 12                      & 15                      & 18                    & 21                      & 24                      & 27 & 30 & \multicolumn{1}{c|}{33} \\
A & \multicolumn{1}{c|}{1}  & 3  &   &   &                       &                         &                         &                       &                         &                         &    &    & \multicolumn{1}{c|}{}   \\
B & \multicolumn{1}{c|}{2}  & 6  &   &   &                       &                         &                         &                       &                         &                         &    &    & \multicolumn{1}{c|}{}   \\ \cline{7-8}
C & \multicolumn{1}{c|}{3}  & 9  &   &   & \multicolumn{1}{c|}{} & \multicolumn{1}{c|}{+0} & \multicolumn{1}{c|}{+3} &                       &                         &                         &    &    & \multicolumn{1}{c|}{}   \\ \cline{7-8}
B & \multicolumn{1}{c|}{4}  & 12 &   &   & \multicolumn{1}{c|}{} & \multicolumn{1}{c|}{+3} & \multicolumn{1}{c|}{}   &                       &                         &                         &    &    & \multicolumn{1}{c|}{}   \\ \cline{7-8}
C & \multicolumn{1}{c|}{5}  & 15 &   &   &                       &                         &                         &                       &                         &                         &    &    & \multicolumn{1}{c|}{}   \\ \cline{10-11}
A & \multicolumn{1}{c|}{6}  & 18 &   &   &                       &                         &                         & \multicolumn{1}{c|}{} & \multicolumn{1}{c|}{+5} & \multicolumn{1}{c|}{+3} &    &    & \multicolumn{1}{c|}{}   \\ \cline{10-11}
B & \multicolumn{1}{c|}{7}  & 21 &   &   &                       &                         &                         & \multicolumn{1}{c|}{} & \multicolumn{1}{c|}{+3} & \multicolumn{1}{c|}{}   &    &    & \multicolumn{1}{c|}{}   \\ \cline{10-11}
B & \multicolumn{1}{c|}{8}  & 24 &   &   &                       &                         &                         &                       &                         &                         &    &    & \multicolumn{1}{c|}{}   \\
A & \multicolumn{1}{c|}{9}  & 27 &   &   &                       &                         &                         &                       &                         &                         &    &    & \multicolumn{1}{c|}{}   \\
C & \multicolumn{1}{c|}{10} & 30 &   &   &                       &                         &                         &                       &                         &                         &    &    & \multicolumn{1}{c|}{}   \\ \cline{3-14}
\end{tabular}

Itt a minimális értéket akarjuk átvenni.
Megállapodás: "óramutató járásával ellentétesen" (egyenlőségkor elsősorban felülről, aztán átlóból, majd vízszintesen)

Átlós út: lecseréljük az egyik betűt a másikra.\\
Függőleges út: beszúrom a másik betűt, törlöm az elsőt.\\
Az átlós út hatékonyabb, mint a függőleges út.

Előadásról:
$
l[i, j] = min
\begin{cases}
    l[i-1, j-1] + c[x_i, y_j]\\
    l[i-1, j] + g\\
    l[i, j-1] + g
\end{cases}
$

Itt még kell valami:
\begin{flalign*}
    & g = 3 &&\\
    & c["p", "q"] = 5 \text{ (a csere 5 költségű)} &&\\
    & c["p", "p"] = 0 \text{ (a nem-csere 0 költségű)} &&
\end{flalign*}

\begin{tabular}{cccccccccccccc}
  &                                                  &                         & B                                              & A                                                                & A                                                                & C                                                                & B                                                                 & B                                                                 & A                                                                 & A                                                                 & C                                                                 & B                                                                 & C                                                                 \\
  &                                                  & 0                       & 1                                              & \fcolorbox{blue}{white}{2}                                       & \fcolorbox{blue}{white}{3}                                       & \fcolorbox{blue}{white}{4}                                       & 5                                                                 & \fcolorbox{blue}{white}{6}                                        & \fcolorbox{blue}{white}{7}                                        & \fcolorbox{blue}{white}{8}                                        & \fcolorbox{blue}{white}{9}                                        & \fcolorbox{blue}{white}{10}                                       & \fcolorbox{blue}{white}{11}                                       \\ \cline{3-14} 
  & \multicolumn{1}{l|}{0}                           & \multicolumn{1}{l|}{0}  & \multicolumn{1}{l|}{\cellcolor[HTML]{34CDF9}3} & \multicolumn{1}{l|}{6}                                           & \multicolumn{1}{l|}{9}                                           & \multicolumn{1}{l|}{12}                                          & \multicolumn{1}{l|}{15}                                           & \multicolumn{1}{l|}{18}                                           & \multicolumn{1}{l|}{21}                                           & \multicolumn{1}{l|}{24}                                           & \multicolumn{1}{l|}{27}                                           & \multicolumn{1}{l|}{30}                                           & \multicolumn{1}{l|}{33}                                           \\ \cline{3-14} 
A & \multicolumn{1}{l|}{\fcolorbox{blue}{white}{1}}  & \multicolumn{1}{l|}{3}  & \multicolumn{1}{l|}{\angledarrow{135} 5}       & \multicolumn{1}{l|}{\cellcolor[HTML]{34CDF9}\angledarrow{135} 3} & \multicolumn{1}{l|}{\angledarrow{135} 6}                         & \multicolumn{1}{l|}{\angledarrow{180} 9}                         & \multicolumn{1}{l|}{\angledarrow{180} 12}                          & \multicolumn{1}{l|}{\angledarrow{180} 15}                         & \multicolumn{1}{l|}{\angledarrow{135} 18}                         & \multicolumn{1}{l|}{\angledarrow{135} 21}                         & \multicolumn{1}{l|}{\angledarrow{180} 24}                         & \multicolumn{1}{l|}{\angledarrow{180} 27}                         & \multicolumn{1}{l|}{\angledarrow{180} 30}                         \\ \cline{3-14} 
B & \multicolumn{1}{l|}{\fcolorbox{blue}{white}{2}}  & \multicolumn{1}{l|}{6}  & \multicolumn{1}{l|}{\angledarrow{135} 3}       & \multicolumn{1}{l|}{\angledarrow{90} 6}                          & \multicolumn{1}{l|}{\cellcolor[HTML]{34CDF9}\angledarrow{135} 8} & \multicolumn{1}{l|}{\angledarrow{135} 11}                        & \multicolumn{1}{l|}{\angledarrow{135} 9}                          & \multicolumn{1}{l|}{\angledarrow{135} 12}                         & \multicolumn{1}{l|}{\angledarrow{180} 15}                         & \multicolumn{1}{l|}{\angledarrow{180} 18}                         & \multicolumn{1}{l|}{\angledarrow{180} 21}                         & \multicolumn{1}{l|}{\angledarrow{135} 24}                         & \multicolumn{1}{l|}{\angledarrow{180} 27}                         \\ \cline{3-14} 
C & \multicolumn{1}{l|}{\fcolorbox{blue}{white}{3}}  & \multicolumn{1}{l|}{9}  & \multicolumn{1}{l|}{\angledarrow{90} 6}        & \multicolumn{1}{l|}{\angledarrow{135} 8}                         & \multicolumn{1}{l|}{\angledarrow{90} 11}                         & \multicolumn{1}{l|}{\cellcolor[HTML]{34CDF9}\angledarrow{135} 8} & \multicolumn{1}{l|}{\cellcolor[HTML]{34CDF9}\angledarrow{180} 11} & \multicolumn{1}{l|}{\angledarrow{135} 14}                         & \multicolumn{1}{l|}{\angledarrow{135} 17}                         & \multicolumn{1}{l|}{\angledarrow{135} 20}                         & \multicolumn{1}{l|}{\angledarrow{135} 18}                         & \multicolumn{1}{l|}{\angledarrow{180} 21}                         & \multicolumn{1}{l|}{\angledarrow{135} 24}                         \\ \cline{3-14} 
B & \multicolumn{1}{l|}{\fcolorbox{blue}{white}{4}}  & \multicolumn{1}{l|}{12} & \multicolumn{1}{l|}{\angledarrow{90} 9}        & \multicolumn{1}{l|}{\angledarrow{90} 11}                         & \multicolumn{1}{l|}{\angledarrow{135} 13}                        & \multicolumn{1}{l|}{\angledarrow{90} 11}                         & \multicolumn{1}{l|}{\angledarrow{135} 8}                          & \multicolumn{1}{l|}{\cellcolor[HTML]{34CDF9}\angledarrow{135} 11} & \multicolumn{1}{l|}{\angledarrow{180} 14}                         & \multicolumn{1}{l|}{\angledarrow{180} 17}                         & \multicolumn{1}{l|}{\angledarrow{180} 20}                         & \multicolumn{1}{l|}{\angledarrow{135} 18}                         & \multicolumn{1}{l|}{\angledarrow{180} 21}                         \\ \cline{3-14} 
C & \multicolumn{1}{l|}{\fcolorbox{blue}{white}{5}}  & \multicolumn{1}{l|}{15} & \multicolumn{1}{l|}{\angledarrow{90} 12}       & \multicolumn{1}{l|}{\angledarrow{90} 14}                         & \multicolumn{1}{l|}{\angledarrow{90} 16}                         & \multicolumn{1}{l|}{\angledarrow{135} 13}                        & \multicolumn{1}{l|}{\angledarrow{90} 11}                          & \multicolumn{1}{l|}{\angledarrow{135} 13}                         & \multicolumn{1}{l|}{\cellcolor[HTML]{34CDF9}\angledarrow{135} 16} & \multicolumn{1}{l|}{\angledarrow{135} 19}                         & \multicolumn{1}{l|}{\angledarrow{135} 17}                         & \multicolumn{1}{l|}{\angledarrow{180} 20}                         & \multicolumn{1}{l|}{\angledarrow{135} 18}                         \\ \cline{3-14} 
A & \multicolumn{1}{l|}{\fcolorbox{blue}{white}{6}}  & \multicolumn{1}{l|}{18} & \multicolumn{1}{l|}{\angledarrow{90} 15}       & \multicolumn{1}{l|}{\angledarrow{135} 12}                        & \multicolumn{1}{l|}{\angledarrow{135} 14}                        & \multicolumn{1}{l|}{\angledarrow{90} 16}                         & \multicolumn{1}{l|}{\angledarrow{90} 14}                          & \multicolumn{1}{l|}{\angledarrow{90} 16}                          & \multicolumn{1}{l|}{\angledarrow{135} 13}                         & \multicolumn{1}{l|}{\cellcolor[HTML]{34CDF9}\angledarrow{135} 16} & \multicolumn{1}{l|}{\angledarrow{180} 19}                         & \multicolumn{1}{l|}{\angledarrow{135} 22}                         & \multicolumn{1}{l|}{\angledarrow{90} 21}                          \\ \cline{3-14} 
B & \multicolumn{1}{l|}{\fcolorbox{blue}{white}{7}}  & \multicolumn{1}{l|}{21} & \multicolumn{1}{l|}{\angledarrow{90} 18}       & \multicolumn{1}{l|}{\angledarrow{90} 15}                         & \multicolumn{1}{l|}{\angledarrow{90} 17}                         & \multicolumn{1}{l|}{\angledarrow{90} 19}                         & \multicolumn{1}{l|}{\angledarrow{135} 16}                         & \multicolumn{1}{l|}{\angledarrow{135} 14}                         & \multicolumn{1}{l|}{\angledarrow{90} 16}                          & \multicolumn{1}{l|}{\angledarrow{135} 18}                         & \multicolumn{1}{l|}{\cellcolor[HTML]{34CDF9}\angledarrow{135} 21} & \multicolumn{1}{l|}{\angledarrow{135}  19}                        & \multicolumn{1}{l|}{\angledarrow{180} 22}                         \\ \cline{3-14} 
B & \multicolumn{1}{l|}{\fcolorbox{blue}{white}{8}}  & \multicolumn{1}{l|}{24} & \multicolumn{1}{l|}{\angledarrow{90} 21}       & \multicolumn{1}{l|}{\angledarrow{90} 18}                         & \multicolumn{1}{l|}{\angledarrow{90} 20}                         & \multicolumn{1}{l|}{\angledarrow{90} 22}                         & \multicolumn{1}{l|}{\angledarrow{90} 19}                          & \multicolumn{1}{l|}{\angledarrow{135} 16}                         & \multicolumn{1}{l|}{\angledarrow{90} 19}                          & \multicolumn{1}{l|}{\angledarrow{90} 21}                          & \multicolumn{1}{l|}{\angledarrow{135} 23}                         & \multicolumn{1}{l|}{\cellcolor[HTML]{34CDF9}\angledarrow{135} 21} & \multicolumn{1}{l|}{\angledarrow{135} 24}                         \\ \cline{3-14} 
A & \multicolumn{1}{l|}{9}                           & \multicolumn{1}{l|}{27} & \multicolumn{1}{l|}{\angledarrow{90} 24}       & \multicolumn{1}{l|}{\angledarrow{90} 21}                         & \multicolumn{1}{l|}{\angledarrow{135} 18}                        & \multicolumn{1}{l|}{\angledarrow{180} 21}                        & \multicolumn{1}{l|}{\angledarrow{90} 22}                          & \multicolumn{1}{l|}{\angledarrow{90} 19}                          & \multicolumn{1}{l|}{\angledarrow{135} 16}                         & \multicolumn{1}{l|}{\angledarrow{135} 19}                         & \multicolumn{1}{l|}{\angledarrow{180} 22}                         & \multicolumn{1}{l|}{\cellcolor[HTML]{34CDF9}\angledarrow{90} 24}  & \multicolumn{1}{l|}{\angledarrow{135} 26}                         \\ \cline{3-14} 
C & \multicolumn{1}{l|}{\fcolorbox{blue}{white}{10}} & \multicolumn{1}{l|}{30} & \multicolumn{1}{l|}{\angledarrow{90} 27}       & \multicolumn{1}{l|}{\angledarrow{90} 24}                         & \multicolumn{1}{l|}{\angledarrow{90} 21}                         & \multicolumn{1}{l|}{\angledarrow{135} 18}                        & \multicolumn{1}{l|}{\angledarrow{180} 21}                         & \multicolumn{1}{l|}{\angledarrow{90} 22}                          & \multicolumn{1}{l|}{\angledarrow{90} 19}                          & \multicolumn{1}{l|}{\angledarrow{135} 21}                         & \multicolumn{1}{l|}{\angledarrow{135} 19}                         & \multicolumn{1}{l|}{\angledarrow{180} 22}                         & \multicolumn{1}{l|}{\cellcolor[HTML]{34CDF9}\angledarrow{135} 24} \\ \cline{3-14} 
\end{tabular}

Optimális szekvenciaillesztés: $\left\{(1, 2), (2, 3), (3, 4), (4, 6), (5, 7), (6, 8), (7, 9), (8, 10), (10, 11)\right\}$

\begin{tabular}{cccccccccccc}
  - & A & B & C & - & B & C & A & B & B & A & C\\
  B & A & A & C & B & B & A & A & C & B & - & C\\
  ins & & rep &  & ins &  & rep &  & rep &  & del & \\
\end{tabular}

(\textit{ins} - beszúrás, \textit{rep} - csere, \textit{del} - törlés)

\textbf{Még egy jópofa sztringes feladat} (egy időben a programozási versenyek egyik kedvence volt)\\
Adott egy $T$ karaktersorozat, keressünk ebben egy leghosszabb olyan karaktersorozatot, amely balról jobbra olvasva ugyanaz, mint jobbról balra olvasva (palindrom)

\sethlcolor{orange}
Pl. T $\longrightarrow$ C {\fcolorbox{blue}{white}{\hl{A}}} D \fcolorbox{blue}{white}{\hl{B}} \hl{C} E \fcolorbox{blue}{white}{\hl{B}} E \fcolorbox{blue}{white}{\hl{A}} B

{\color{blue} Ez egy részsorozat, de nem a leghosszabb.}
{\color{orange} Ez is egy részsorozat, a leghosszabb.}

$T$-ből készítsünk palindromot a legkevesebb karakter törlésével!

Hasonló feladat: $T$-ből készítsünk palindromot a legkevesebb karakter beszúrásával!