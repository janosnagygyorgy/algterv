\section{gyakorlat (2025. november 19.)}
\subsection{Hátizsák feladat}
\sethlcolor{orange}
\textbf{Példa} \hl{(zh. feladat)}:\\
\begin{tabular}{c|c|c|c|c|c|c|c|}
\cline{2-8}
i     & 1  & 2  & 3  & 4  & 5 & 6  & 7  \\ \cline{2-8} 
$w_i$ & 4  & 5  & 3  & 1  & 1 & 4  & 3  \\ \cline{2-8} 
$v_i$ & 40 & 60 & 10 & 10 & 3 & 20 & 60 \\ \cline{2-8} 
\end{tabular}

$W = 10$ (hátizsák kapacitása)

Részfeladatok: az első néhány tárggyal kell megtölteni optimálisan kisebb hátizsákokat.

Kitöltés menete:\\
Az oszlop a hátizsák kapacitása, a sor a hátizsákba rakott érték.
Minden sorhoz tartozik egy tárgy (az $i.$ sorhoz $t_i$ és $v_i$) egy \sethlcolor{red}\hl{súly}/\sethlcolor{yellow}\hl{érték} párral.
Megnézzük milyen \sethlcolor{lime}\hl{érték} van a jelenlegi cellától eggyel fentebbi cellában.
Visszalépünk abban a sorban \sethlcolor{red}\hl{súly} cellát.
Amelyik cellához \sethlcolor{pink}\hl{jutottunk}, azt hozzáadjuk az \sethlcolor{yellow}\hl{értékhez}.
Az összeg és a cella feletti \sethlcolor{lime}\hl{érték} közül a nagyobb lesz a cella tartalma.

Előadásról:
$
h[i, j] = max
\begin{cases}
    v_i + h[i-1, j-w_i]\\
    h[i-1, j]\\
\end{cases}
$

Ha pl. az sorhoz tartozó tárgy súlya 4, akkor a 0, 1, 2, 3 indexű oszlopok csak másolódnak, mert nem tudunk 4-et visszalépni a felette lévő sorban.
Azaz, 0, 1, 2, 3 kapacitású hátizsákba nem fér bele a 4 súlyú tárgy.
Minden sorban azt nézzük, hogy a sorhoz tartozó tárgyat érdemes-e belerakni a hátizsákba.

\sethlcolor{yellow}
\begin{tabular}{ccccccccccccc}
     & i\textbackslash{}j                             & 0                                              & 1                       & 2                       & 3                       & 4                       & 5                                               & 6                                               & 7                                               & 8                        & 9                              & 10                                               \\ \cline{3-13} 
     & \multicolumn{1}{c|}{0}                         & \multicolumn{1}{c|}{0}                         & \multicolumn{1}{c|}{0}  & \multicolumn{1}{c|}{0}  & \multicolumn{1}{c|}{0}  & \multicolumn{1}{c|}{0}  & \multicolumn{1}{c|}{0}                          & \multicolumn{1}{c|}{0}                          & \multicolumn{1}{c|}{0}                          & \multicolumn{1}{c|}{0}   & \multicolumn{1}{c|}{0}         & \multicolumn{1}{c|}{0}                           \\ \cline{3-13} 
4/40 & \multicolumn{1}{c|}{1}                         & \multicolumn{1}{c|}{\cellcolor[HTML]{34CDF9}0} & \multicolumn{1}{c|}{0}  & \multicolumn{1}{c|}{0}  & \multicolumn{1}{c|}{0}  & \multicolumn{1}{c|}{\sethlcolor{pink}\hl{40}} & \multicolumn{1}{c|}{40}                         & \multicolumn{1}{c|}{40}                         & \multicolumn{1}{c|}{40}                         & \multicolumn{1}{c|}{40}  & \multicolumn{1}{c|}{\sethlcolor{lime}\hl{40}}        & \multicolumn{1}{c|}{40}                          \\ \cline{3-13} 
\sethlcolor{red}\hl{5}/\sethlcolor{yellow}\hl{60} & \multicolumn{1}{c|}{\fcolorbox{blue}{white}{2}} & \multicolumn{1}{c|}{0}                         & \multicolumn{1}{c|}{0}  & \multicolumn{1}{c|}{0}  & \multicolumn{1}{c|}{0}  & \multicolumn{1}{c|}{40} & \multicolumn{1}{c|}{\cellcolor[HTML]{34CDF9}60} & \multicolumn{1}{c|}{60}                         & \multicolumn{1}{c|}{60}                         & \multicolumn{1}{c|}{60}  & \multicolumn{1}{c|}{max(\sethlcolor{lime}\hl{40}, (\sethlcolor{pink}\hl{40}+\sethlcolor{yellow}\hl{60}))=100} & \multicolumn{1}{c|}{100}                         \\ \cline{3-13} 
3/10 & \multicolumn{1}{c|}{3}                         & \multicolumn{1}{c|}{0}                         & \multicolumn{1}{c|}{0}  & \multicolumn{1}{c|}{0}  & \multicolumn{1}{c|}{10} & \multicolumn{1}{c|}{40} & \multicolumn{1}{c|}{\cellcolor[HTML]{34CDF9}60} & \multicolumn{1}{c|}{60}                         & \multicolumn{1}{c|}{60}                         & \multicolumn{1}{c|}{70}  & \multicolumn{1}{c|}{100}       & \multicolumn{1}{c|}{100}                         \\ \cline{3-13} 
1/10 & \multicolumn{1}{c|}{\fcolorbox{blue}{white}{4}} & \multicolumn{1}{c|}{0}                         & \multicolumn{1}{c|}{10} & \multicolumn{1}{c|}{10} & \multicolumn{1}{c|}{10} & \multicolumn{1}{c|}{40} & \multicolumn{1}{c|}{60}                         & \multicolumn{1}{c|}{\cellcolor[HTML]{34CDF9}70} & \multicolumn{1}{c|}{70}                         & \multicolumn{1}{c|}{70}  & \multicolumn{1}{c|}{100}       & \multicolumn{1}{c|}{110}                         \\ \cline{3-13} 
1/3  & \multicolumn{1}{c|}{\fcolorbox{blue}{white}{5}} & \multicolumn{1}{c|}{0}                         & \multicolumn{1}{c|}{10} & \multicolumn{1}{c|}{13} & \multicolumn{1}{c|}{13} & \multicolumn{1}{c|}{40} & \multicolumn{1}{c|}{60}                         & \multicolumn{1}{c|}{70}                         & \multicolumn{1}{c|}{\cellcolor[HTML]{34CDF9}73} & \multicolumn{1}{c|}{73}  & \multicolumn{1}{c|}{100}       & \multicolumn{1}{c|}{110}                         \\ \cline{3-13} 
4/20 & \multicolumn{1}{c|}{6}                         & \multicolumn{1}{c|}{0}                         & \multicolumn{1}{c|}{10} & \multicolumn{1}{c|}{13} & \multicolumn{1}{c|}{13} & \multicolumn{1}{c|}{40} & \multicolumn{1}{c|}{60}                         & \multicolumn{1}{c|}{70}                         & \multicolumn{1}{c|}{\cellcolor[HTML]{34CDF9}73} & \multicolumn{1}{c|}{73}  & \multicolumn{1}{c|}{100}       & \multicolumn{1}{c|}{110}                         \\ \cline{3-13} 
3/60 & \multicolumn{1}{c|}{\fcolorbox{blue}{white}{7}} & \multicolumn{1}{c|}{0}                         & \multicolumn{1}{c|}{10} & \multicolumn{1}{c|}{13} & \multicolumn{1}{c|}{60} & \multicolumn{1}{c|}{70} & \multicolumn{1}{c|}{73}                         & \multicolumn{1}{c|}{73}                         & \multicolumn{1}{c|}{100}                        & \multicolumn{1}{c|}{120} & \multicolumn{1}{c|}{130}       & \multicolumn{1}{c|}{\cellcolor[HTML]{34CDF9}133} \\ \cline{3-13} 
\end{tabular}

Eredmény leolvasása:\\
A jobb alsó sarokból indulunk szokás szerint.
A 7. sorban még nőtt az érték, tehát a 7. tárgyra szükség van, ezt "beragasztjuk" a hátizsákba.
Ekkor már beragaszottunk 3 kapacitásnyi tárgyat a 10 kapacitású hátizsákba.
Az első 7 kapacitást kell még valahogy kitölteni.
Ezért nézzük a 7. oszlopot és felfelé haladva a következő sort.
Ez a cella az előző sorhoz képest nem javult, azaz a 6. tárgyra nincs szükség.
Megyünk megint egy sorral fentebb, ugyanabban az oszlopban.
Ez javított az őt megelőző sorhoz képest, ezért erre a tárgyra szükség van.
Ezt ismételjük tovább.

Optimális megoldás: $t_2, t_4, t_5, t_7$